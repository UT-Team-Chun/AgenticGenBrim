\documentclass[11pt,a4paper]{ltjsarticle}

% Avoid font size substitution warnings for very small font sizes
\usepackage{type1cm}

\usepackage[a4paper,margin=25mm]{geometry}
\usepackage{hyperref}
\usepackage{bookmark}
\usepackage{fancyhdr}
\usepackage{enumitem}
\usepackage{longtable}

% Diagrams
\usepackage{graphicx}
\usepackage{tikz}
\usetikzlibrary{arrows.meta,positioning,fit,calc}

% For preformatted blocks
\usepackage{fvextra}
\DefineVerbatimEnvironment{verbatim}{Verbatim}{breaklines=true,breakanywhere=true,fontsize=\small}

\renewcommand{\contentsname}{目次}
\setcounter{tocdepth}{2}

\pagestyle{fancy}
\fancyhf{}
\lhead{}
\rhead{\nouppercase{\leftmark}}
\cfoot{\thepage}

\setlist[itemize]{leftmargin=2.0em}
\setlist[enumerate]{leftmargin=2.2em}

\title{鋼プレートガーダー橋 BrIM 生成エージェントシステム\\{\large ― 現状の機能概要と技術仕様 ―}}
\author{阿部大樹}
\date{2026年1月7日}

\begin{document}
\maketitle
\thispagestyle{empty}
\clearpage
\setcounter{page}{1}

  \tableofcontents
\clearpage

\section{概要}



\subsection{本システムでできること}


\begin{enumerate}
  \item 最小限の入力から断面設計を自動生成
\begin{itemize}
  \item 入力:橋長 L [m] と 幅員 B [m] の 2 パラメータのみ
  \item 出力:主桁本数、桁高、板厚、床版厚など全断面寸法を自動決定
  \par
\end{itemize}
  \item 設計根拠の明示(RAG による文献参照)
\begin{itemize}
  \item 道路橋示方書、鋼橋設計の基本(教科書)から関連条文を自動検索
  \item どの文献のどのページを根拠にしたかをログに記録
  \item 設計者が「なぜその寸法になったか」を確認可能
  \par
\end{itemize}
  \item 設計ルールの構造化抽出
\begin{itemize}
  \item 「桁高 ≒ L/20」「床版厚 d = 30L + 110」などの数式を明示
  \item 各ルールに根拠(文献の rank)または「仮定」を付与
  \item 将来の設計ルールDB構築の基盤
  \par
\end{itemize}
  \item IFC 形式での 3D モデル出力
\begin{itemize}
  \item BIM/CIM ソフトウェアで表示可能な IFC ファイルを生成
  \item 床版、主桁(I形断面)、横桁を 3D モデル化
  \item BIMvision、Revit、FreeCAD 等で即座に可視化可能
  \par
  \par
\end{itemize}
\end{enumerate}

\begin{figure}[ht]
\centering
\includegraphics[width=0.92\linewidth]{images/bim_model.png}
\caption{BIM/CIM ビューアで表示した IFC モデル}
\label{fig:bim-model}
\end{figure}


\subsubsection{実証済みの成果}



\begin{table}[ht]
\centering
\caption{生成実績}
\begin{tabular}{ll}
\hline
項目 & 値 \\
\hline
設計 JSON 生成数 & 26 件 \\
IFC ファイル生成数 & 21 件 \\
対象橋長 & 30m ~ 50m \\
対象幅員 & 5m ~ 10m \\
テスト期間 & 2025年12月 ~ 2026年1月 \\
\hline
\end{tabular}
\end{table}

\begin{table}[ht]
\centering
\caption{設計精度の検証(橋長50m、幅員10m の例)}
\begin{tabular}{lll}
\hline
項目 & 生成値 & 評価 \\
\hline
桁高/橋長比 & 1/19.3 & 文献目安 L/20 に合致 \\
床版厚 & 190mm & 連続版式 $d=30L+110$ に合致 \\
横桁高/主桁高比 & 0.8 & 文献の経験則に合致 \\
主桁間隔 & 2.67m & 実務例 3.36m と同オーダー \\
\hline
\end{tabular}
\end{table}


\paragraph{技術的な特徴}


\begin{itemize}
  \item LLM + RAG による設計知識の活用
  \\ → 道路橋示方書等の PDF から条文を検索し、プロンプトに埋め込み
  \par
  \item Structured Output による型安全な出力
  \\ → Pydantic スキーマで設計値を厳密に定義、バリデーション済み
  \par
  \item マルチクエリ RAG
  \\ → 寸法、主桁配置、主桁断面、床版、横桁の 5 観点で並行検索
  \par
  \item 2段階の IFC 変換
  \\ → 設計JSON → 詳細JSON → IFC の変換パイプライン
  \par
  \par
\end{itemize}


\paragraph{現時点での制約}


\begin{itemize}
  \item 対象は鋼プレートガーダー橋(RC床版)のみ
  \item 断面は全長一定(支点部・中央部の変化なし)
  \item 荷重・応力照査は未実装(概略設計レベル)
  \item Judge(設計評価)はダミー実装
  \par
  \par
\end{itemize}


\section{序論}



\subsection{背景と目的}


土木インフラ分野における BIM/CIM(Building/Construction Information Modeling)の導入が進む中、橋梁設計の初期段階における概略設計の自動化・効率化が求められている。本システムは、大規模言語モデル(LLM)と検索拡張生成(RAG)技術を活用し、鋼プレートガーダー橋の断面設計を自動生成するエージェント型システムである。


本システムの目的は以下の通りである:


\begin{enumerate}
  \item 橋長と幅員という最小限の入力から、道路橋示方書等に基づいた妥当な
  \\ 断面寸法の初期案を自動生成する
  \par
  \item 設計根拠となる条文・数式を明示し、設計プロセスの透明性を確保する
  \par
  \item 生成された設計情報を IFC(Industry Foundation Classes)形式で出力し、
  \\ BIM/CIM 環境との連携を実現する
  \par
  \par
\end{enumerate}


\subsection{システム概要}


本システム「AgenticGenBrim」は、以下の主要コンポーネントから構成される:


\begin{itemize}
  \item RAG サブシステム:設計知識(PDF文書)の検索・参照機能
  \item Designer サブシステム:LLM による橋梁断面設計の生成
  \item IFC 変換サブシステム:設計 JSON から IFC ファイルへの変換
  \item 統合 CLI:上記コンポーネントを一括実行するインターフェース
  \par
\end{itemize}

システムは Python 3.13 で実装され、OpenAI API を活用したStructured Output(構造化出力)により、型安全な設計データを生成する。




\subsection{対象とする橋梁形式}


本 MVP では、以下の橋梁形式を対象とする:


\begin{itemize}
  \item 橋梁タイプ:鋼プレートガーダー橋(Steel Plate Girder Bridge)
  \item 床版形式:RC 床版合成桁(Reinforced Concrete Deck Composite Girder)
  \item 構造形式:単純桁または連続桁(MVP では単純桁を想定)
  \par
\end{itemize}

対象とする設計パラメータ:

\begin{itemize}
  \item 橋長(L):30m ~ 100m 程度
  \item 幅員(B):8m ~ 15m 程度
  \par
  \par
\end{itemize}


\clearpage

\section{システムアーキテクチャ}



\subsection{全体構成}


システムは以下の 3 層構造で構成される:



\paragraph{ユーザーインターフェース層}
\begin{description}
  \item[統合 CLI] \texttt{src/main.py}
  \begin{itemize}
    \item \texttt{generate} コマンド:設計 JSON 生成
    \item \texttt{run} コマンド:設計 JSON 生成 → IFC 変換まで一括実行
  \end{itemize}
\end{description}

\paragraph{コアロジック層}
\begin{description}
  \item[RAG サブシステム] ~
  \begin{itemize}
    \item PDF テキスト抽出
    \item チャンク化
    \item 埋め込み生成
    \item ベクトル検索
  \end{itemize}
  \item[Designer サブシステム] ~
  \begin{itemize}
    \item プロンプト構築
    \item LLM 呼び出し
    \item 設計生成
  \end{itemize}
  \item[IFC 変換サブシステム] ~
  \begin{itemize}
    \item Simple → Detailed 変換
    \item Detailed → IFC 変換
  \end{itemize}
\end{description}

\paragraph{外部サービス層}
\begin{description}
  \item[OpenAI API] ~
  \begin{itemize}
    \item Responses API(テキスト生成)
    \item Embeddings API(ベクトル埋め込み)
    \item Structured Output(構造化出力)
  \end{itemize}
\end{description}

\begin{figure}[ht]
\centering
\resizebox{\linewidth}{!}{%
\begin{tikzpicture}[
  font=\small,
  box/.style={draw,rounded corners,align=center,inner sep=4pt,minimum height=8mm,text width=26mm},
  storebox/.style={box,text width=34mm},
  lab/.style={font=\scriptsize,fill=white,inner sep=1.5pt},
  arrow/.style={-Latex,thick},
  node distance=10mm and 14mm
]
  \node[box] (input) {ユーザー入力\\橋長 $L$\\幅員 $B$};
  \node[box, right=of input] (cli) {統合CLI\\(generate/run)};
  \node[box, right=of cli] (retrieve) {Retrieve\\(検索・参照)};
  \node[box, right=of retrieve] (designer) {Designer\\(設計生成)};
  \node[box, right=of designer] (ifc) {IFC変換\\(JSON→IFC)};
  \node[box, right=of ifc] (viewer) {BIM/CIM\\ビューア};

  \node[storebox, below=16mm of designer] (store) {出力\\設計JSON\\詳細JSON\\IFC\\RAGログ};

  \node[box, above=14mm of retrieve] (pdf) {PDF知識ベース\\(示方書等)};
  \node[box, above=14mm of designer] (openai) {OpenAI API\\Embeddings\\LLM};

  \draw[arrow] (input) -- (cli);
  \draw[arrow] (cli) -- (retrieve);
  \draw[arrow] (retrieve) -- node[midway, above, yshift=5pt, lab]{参考文献コンテキスト} (designer);
  \draw[arrow] (designer) -- (ifc);
  \draw[arrow] (ifc) -- (viewer);

  \draw[arrow] (designer) -- (store);
  \draw[arrow] (ifc) -- (store);

  \draw[arrow] (pdf) -- (retrieve);
  \draw[arrow] (retrieve) -- node[midway, above left, xshift=-2pt, yshift=3pt, lab]{埋め込み} (openai);
  \draw[arrow] (designer) -- node[midway, above right, xshift=2pt, yshift=3pt, lab]{Structured Output} (openai);
\end{tikzpicture}
}%
\caption{システム全体構成}
\label{fig:system-block}
\end{figure}


\subsection{データフロー}



\subsubsection{設計生成から IFC 出力までの処理フロー}



\noindent\textbf{入力:} 橋長 $L$ [m]、幅員 $B$ [m]

\begin{enumerate}
  \item \textbf{マルチクエリ RAG 検索}
  \begin{itemize}
    \item 寸法関連クエリ → \texttt{dimensions\_context}
    \item 主桁配置クエリ → \texttt{girder\_layout\_context}
    \item 主桁断面クエリ → \texttt{girder\_section\_context}
    \item 床版クエリ → \texttt{deck\_context}
    \item 横桁クエリ → \texttt{crossbeam\_context}
  \end{itemize}

  \item \textbf{プロンプト構築}
  \begin{itemize}
    \item RAG コンテキストを参考文献として埋め込み
    \item 設計手順と制約条件を明示
  \end{itemize}

  \item \textbf{LLM 呼び出し(Structured Output)}
  \begin{itemize}
    \item \texttt{DesignerOutput} スキーマに従った構造化出力
    \item \texttt{reasoning}(設計根拠)+ \texttt{rules}(設計ルール)+ \texttt{bridge\_design}(設計値)
  \end{itemize}

  \item \textbf{BridgeDesign JSON 出力}
  \begin{itemize}
    \item \texttt{data/generated\_simple\_bridge\_json/} に保存
    \item RAG ログを \texttt{data/generated\_bridge\_raglog\_json/} に保存
  \end{itemize}

  \item \textbf{Simple JSON → Detailed JSON 変換}
  \begin{itemize}
    \item 座標計算(主桁配置、格間分割)
    \item IFC 生成用の詳細ジオメトリ情報を追加
  \end{itemize}

  \item \textbf{Detailed JSON → IFC 変換}
  \begin{itemize}
    \item ifcopenshell による IFC 要素生成
    \item 床版(Brep)、主桁(SweptSolid)、横桁(SweptSolid)
  \end{itemize}
\end{enumerate}

\noindent\textbf{出力:} IFC ファイル


\clearpage

\section{RAG(検索拡張生成)サブシステム}



\subsection{概要}


RAG(Retrieval-Augmented Generation)サブシステムは、道路橋示方書や鋼橋設計の教科書から関連する条文・解説を検索し、Designer の LLM プロンプトに参考文献として提供する機能を担う。


主な機能:

\begin{itemize}
  \item PDF ドキュメントからのテキスト抽出
  \item テキストのチャンク化(分割)
  \item OpenAI Embeddings API による埋め込みベクトル生成
  \item コサイン類似度に基づくベクトル検索
  \par
  \par
\end{itemize}


\subsection{対象ドキュメント}


現在 RAG で利用している PDF ドキュメント(FileNamesUsedForRag で定義):


1. 鋼橋設計の基本\_第一章 概論.pdf

\begin{itemize}
  \item 鋼橋の基本概念、設計の考え方
  \par
\end{itemize}

2. 鋼橋設計の基本\_第四章 鋼橋の設計法.pdf

\begin{itemize}
  \item 鋼橋の設計手法、荷重・応力計算
  \par
\end{itemize}

3. 鋼橋設計の基本\_第六章 床版.pdf

\begin{itemize}
  \item RC 床版の設計、厚さ算定式
  \par
\end{itemize}

4. 鋼橋設計の基本\_第七章 プレートガーダー橋.pdf

\begin{itemize}
  \item プレートガーダー橋の構造・設計
  \par
\end{itemize}

5. 道路橋示方書\_鋼橋・鋼部材編.pdf

\begin{itemize}
  \item 公式の設計基準、条文
  \par
  \par
\end{itemize}


\subsection{インデックス構築と検索}

PDF からのテキスト抽出には pdfplumber を使用し、以下の処理を行う:

\begin{enumerate}
  \item PDF からテキストを抽出し、ページ境界を記録
  \item 各ページを最大 800 文字ごとにチャンク化
  \item OpenAI の text-embedding-3-small で埋め込みベクトルを生成
  \item クエリとのコサイン類似度に基づき関連チャンクを検索
\end{enumerate}

\begin{figure}[ht]
\centering
\resizebox{\linewidth}{!}{%
\begin{tikzpicture}[
  font=\small,
  box/.style={draw,rounded corners,align=center,inner sep=4pt,minimum height=8mm,text width=32mm},
  arrow/.style={-Latex,thick},
  node distance=7mm and 18mm
]
  \node[box] (pdf) {PDF文書};
  \node[box, below=of pdf] (extract) {テキスト抽出\\(pdfplumber)};
  \node[box, below=of extract] (chunk) {チャンク化\\(最大800文字)};
  \node[box, below=of chunk] (emb) {埋め込み生成\\(Embeddings)};
  \node[box, below=of emb] (index) {ベクトル\\インデックス};

  \node[box, right=62mm of extract] (query) {ユーザークエリ};
  \node[box, below=of query] (qemb) {クエリ埋め込み};
  \node[box, below=of qemb] (search) {類似度検索\\(cosine, top\_k)};
  \node[box, below=of search] (chunks) {上位チャンク\\(本文+出典+score)};
  \node[box, below=of chunks] (prompt) {LLMプロンプト\\へ挿入};

  \draw[arrow] (pdf) -- (extract);
  \draw[arrow] (extract) -- (chunk);
  \draw[arrow] (chunk) -- (emb);
  \draw[arrow] (emb) -- (index);

  \draw[arrow] (query) -- (qemb);
  \draw[arrow] (qemb) -- (search);
  \draw[arrow] (search) -- (chunks);
  \draw[arrow] (chunks) -- (prompt);

  \draw[arrow] (index.east) -- (search.west);

  \node[draw,dashed,rounded corners,fit=(pdf)(index),inner sep=8pt,label={[font=\scriptsize]above:{事前処理(インデックス構築)}}] {};
  \node[draw,dashed,rounded corners,fit=(query)(prompt),inner sep=8pt,label={[font=\scriptsize]above:{検索時(クエリ実行)}}] {};
\end{tikzpicture}
}%
\caption{RAG の検索パイプライン(概略)}
\label{fig:rag-pipeline}
\end{figure}


\clearpage

\section{Designer(設計生成)サブシステム}



\subsection{概要}


Designer サブシステムは、橋長と幅員を入力として受け取り、RAG で取得した参考文献を基に LLM(GPT-5-mini / GPT-5.1)を用いて鋼プレートガーダー橋の断面設計を自動生成する。


主な特徴:

\begin{itemize}
  \item Structured Output による型安全な設計データ生成
  \item マルチクエリ RAG による多面的な参考文献検索
  \item 設計ルール(DesignRule)の明示的抽出
  \item 設計根拠(reasoning)の記録
  \par
  \par
\end{itemize}


\subsection{入力仕様}



\subsubsection{DesignerInput スキーマ}

\begin{verbatim}
class DesignerInput(BaseModel):
  bridge_length_m: float  # 橋長 L [m]
  total_width_m: float    # 幅員 B [m]
\end{verbatim}


入力パラメータは最小限の 2 つのみとし、その他の設計パラメータ(主桁本数、桁高、板厚など)は LLM が RAG コンテキストを参照して自動決定する。




\subsection{出力仕様(BridgeDesign スキーマ)}



\subsubsection{BridgeDesign 構造}


\begin{description}
  \item[\texttt{BridgeDesign}] トップレベルモデル
  \begin{description}
    \item[\texttt{dimensions}: Dimensions] 全体寸法
    \begin{itemize}
      \item \texttt{bridge\_length}: float --- 橋長 [mm]
      \item \texttt{total\_width}: float --- 全幅 [mm]
      \item \texttt{num\_girders}: int --- 主桁本数
      \item \texttt{girder\_spacing}: float --- 主桁間隔 [mm]
      \item \texttt{panel\_length}: float --- パネル長 [mm]
      \item \texttt{num\_panels}: int | None --- パネル数(自動計算可)
    \end{itemize}
    \item[\texttt{sections}: Sections] 断面情報
    \begin{description}
      \item[\texttt{girder\_standard}: GirderSection] 主桁標準断面
      \begin{itemize}
        \item \texttt{web\_height}: float --- 腹板高さ [mm]
        \item \texttt{web\_thickness}: float --- 腹板厚 [mm]
        \item \texttt{top\_flange\_width}: float --- 上フランジ幅 [mm]
        \item \texttt{top\_flange\_thickness}: float --- 上フランジ厚 [mm]
        \item \texttt{bottom\_flange\_width}: float --- 下フランジ幅 [mm]
        \item \texttt{bottom\_flange\_thickness}: float --- 下フランジ厚 [mm]
      \end{itemize}
      \item[\texttt{crossbeam\_standard}: CrossbeamSection] 横桁標準断面
      \begin{itemize}
        \item \texttt{total\_height}: float --- 桁高 [mm]
        \item \texttt{web\_thickness}: float --- 腹板厚 [mm]
        \item \texttt{flange\_width}: float --- フランジ幅 [mm]
        \item \texttt{flange\_thickness}: float --- フランジ厚 [mm]
      \end{itemize}
    \end{description}
    \item[\texttt{components}: Components] 構成要素
    \begin{description}
      \item[\texttt{deck}: Deck] RC 床版
      \begin{itemize}
        \item \texttt{thickness}: float --- 床版厚 [mm]
      \end{itemize}
    \end{description}
  \end{description}
\end{description}

\paragraph{DesignerOutput 構造}
LLM からの直接出力スキーマ:

\begin{description}
  \item[\texttt{DesignerOutput}] ~
  \begin{itemize}
    \item \texttt{reasoning}: str --- 設計プロセスの思考・判断根拠
    \item \texttt{rules}: list[DesignRule] --- 適用した設計ルール一覧
    \item \texttt{bridge\_design}: BridgeDesign --- 生成された設計
  \end{itemize}
\end{description}

\paragraph{DesignRule 構造}

\begin{description}
  \item[\texttt{DesignRule}] ~
  \begin{itemize}
    \item \texttt{rule\_id}: str --- ルールID("R1", "R2" など)
    \item \texttt{category}: DesignRuleCategory --- カテゴリ(dimensions/girder\_section/deck/crossbeam\_section/other)
    \item \texttt{summary}: str --- ルール内容の日本語要約
    \item \texttt{condition\_expression}: str | None --- 条件式(例:\texttt{web\_height $\approx$ L/20\textasciitilde L/25})
    \item \texttt{formula\_latex}: str | None --- 数式の LaTeX 表現
    \item \texttt{applies\_to\_fields}: list[str] --- 影響するフィールド名
    \item \texttt{source\_hit\_ranks}: list[int] --- 根拠となる RAG ヒットの rank
    \item \texttt{notes}: str | None --- 補足・注意事項
  \end{itemize}
\end{description}


\subsection{マルチクエリ RAG 連携}


Designer は設計生成時に、以下の 5 種類のクエリで RAG 検索を実行する:


\begin{enumerate}
  \item 寸法関連(dimensions)
  \\ クエリ: "鋼プレートガーダー橋 橋長\{L\}m 幅員\{B\}m 桁配置 主桁本数 桁間隔 パネル長"
  \par
  \item 主桁配置(girder\_layout)
  \\ クエリ: "並列I桁 主桁間隔 幅員と主桁本数の関係 標準断面 主桁本数"
  \par
  \item 主桁断面(girder\_section)
  \\ クエリ: "プレートガーダー橋 橋長\{L\}m 主桁断面 桁高 腹板厚さ フランジ幅 フランジ厚さ 経済的桁高 h/L"
  \par
  \item RC 床版(deck)
  \\ クエリ: "RC床版合成桁 床版厚さ 最小床版厚 床版厚と支間の比"
  \par
  \item 横桁(crossbeam)
  \\ クエリ: "横桁 対傾構 横構 設計"
  \par
\end{enumerate}

各クエリで top\_k 件(デフォルト 5 件)を取得し、合計最大 25 チャンクの参考文献をプロンプトに含める。

\begin{figure}[ht]
\centering
\resizebox{\linewidth}{!}{%
\begin{tikzpicture}[
  font=\small,
  box/.style={draw,rounded corners,align=center,inner sep=4pt,minimum height=8mm,text width=34mm},
  arrow/.style={-Latex,thick},
  node distance=11mm and 16mm
]
  \node[box, text width=20mm] (inp) {入力\\$L, B$};
  \node[box, right=18mm of inp] (q1) {寸法\\(dimensions)};
  \node[box, below=of q1] (q2) {主桁配置\\\texttt{girder\_layout}};
  \node[box, below=of q2] (q3) {主桁断面\\\texttt{girder\_section}};
  \node[box, below=of q3] (q4) {床版\\\texttt{deck}};
  \node[box, below=of q4] (q5) {横桁\\\texttt{crossbeam}};

  \node[box, right=18mm of q3, text width=30mm] (rag) {RAG検索\\(各 top\_k=5)};
  \node[box, right=18mm of rag, text width=34mm] (ctx) {統合RAG\\コンテキスト\\(最大25\\チャンク)};
  \node[box, right=18mm of ctx, text width=26mm] (llm) {Designer\\(LLM)};
  \node[box, right=14mm of llm, text width=24mm] (out) {設計JSON\\+ログ};

  % Spread outgoing arrows from the input box and aim at the middle of each target's left edge.
  \coordinate (inpa) at ($(inp.north east)!0.15!(inp.south east)$);
  \coordinate (inpb) at ($(inp.north east)!0.32!(inp.south east)$);
  \coordinate (inpc) at ($(inp.north east)!0.50!(inp.south east)$);
  \coordinate (inpd) at ($(inp.north east)!0.68!(inp.south east)$);
  \coordinate (inpe) at ($(inp.north east)!0.85!(inp.south east)$);

  \draw[arrow] (inpa) |- (q1.west);
  \draw[arrow] (inpb) |- (q2.west);
  \draw[arrow] (inpc) |- (q3.west);
  \draw[arrow] (inpd) |- (q4.west);
  \draw[arrow] (inpe) |- (q5.west);

  \draw[arrow] (q1) -- (rag);
  \draw[arrow] (q2) -- (rag);
  \draw[arrow] (q3) -- (rag);
  \draw[arrow] (q4) -- (rag);
  \draw[arrow] (q5) -- (rag);

  \draw[arrow] (rag) -- (ctx);
  \draw[arrow] (ctx) -- (llm);
  \draw[arrow] (llm) -- (out);
\end{tikzpicture}
}%
\caption{マルチクエリ RAG(5観点)と Designer への統合}
\label{fig:multiquery-rag}
\end{figure}




\subsection{プロンプトエンジニアリング}


Designer プロンプトの主要構成:


\begin{enumerate}
  \item ロール設定
  \\ 「あなたは鋼橋設計の専門家です。」
  \par
  \item 入力条件
  \\ 橋長 L、幅員 B、橋種(鋼プレートガーダー橋 RC床版合成桁)
  \par
  \item 参考文献(RAG Context)
\begin{itemize}
  \item [1] 桁配置・支間割・全体諸元
  \item [2] 主桁配置(桁本数・主桁間隔)
  \item [3] 主桁断面
  \item [4] RC床版
  \item [5] 横桁・床桁
  \item [6] その他
  \par
\end{itemize}
  \item 設計手順と注意事項
\begin{itemize}
  \item 思考プロセスの記述(reasoning)
  \item 設計ルールの抽出(rules)
  \item 断面諸元の決定(bridge\_design)
  \par
\end{itemize}
  \item 重要な制約
\begin{itemize}
  \item 幾何学的整合性(全幅 = (主桁本数-1)×桁間隔 + 2×張出し長)
  \item 主桁本数の候補比較(3〜6本)
  \item パネル長の整数割り当て
  \item すべての寸法は mm 単位
  \par
  \par
\end{itemize}
\end{enumerate}


\subsection{設計ルール抽出機能}


Designer は設計値だけでなく、適用した設計ルールを明示的に抽出する。



\subsubsection{DesignRuleCategory}


\begin{itemize}
  \item dimensions: 橋長・幅員・桁本数・桁間隔・パネル長など全体寸法
  \item girder\_section: 主桁断面
  \item deck: RC床版
  \item crossbeam\_section: 横桁
  \item other: その他
  \par
\end{itemize}


\paragraph{source\_hit\_ranks の取り扱い}


\begin{itemize}
  \item RAG ヒットに根拠がある場合:該当する rank 番号をリストで記載
  \item 根拠が曖昧/見当たらない場合:source\_hit\_ranks: [] とし、
  \\ notes に「仮定」「実務目安」等を明記
  \par
  \par
\end{itemize}


\clearpage

\section{IFC 変換サブシステム}



\subsection{概要}


IFC 変換サブシステムは、Designer が生成した BridgeDesign JSON をIFC(Industry Foundation Classes)形式に変換し、BIM/CIM ソフトウェアで利用可能な 3D モデルを出力する。


変換は 2 段階で行われる:

\begin{enumerate}
  \item BridgeDesign JSON → 詳細 JSON(DetailedBridgeJson)
  \item 詳細 JSON → IFC ファイル
  \par
  \par
\end{enumerate}

\begin{figure}[ht]
\centering
\begin{tikzpicture}[
  font=\small,
  box/.style={draw,rounded corners,align=center,inner sep=4pt,minimum width=26mm,minimum height=9mm,text width=26mm},
  arrow/.style={-Latex,thick},
  node distance=18mm
]
  \node[box] (simple) {BridgeDesign\\JSON};
  \node[box, right=of simple] (detailed) {Detailed\\JSON};
  \node[box, right=of detailed] (ifc) {IFC};
  \draw[arrow] (simple) -- node[above, align=center, font=\scriptsize]{座標付与・分割} (detailed);
  \draw[arrow] (detailed) -- node[above, align=center, font=\scriptsize]{ifcopenshell} (ifc);
\end{tikzpicture}
\caption{IFC 変換の 2 段階パイプライン}
\label{fig:ifc-pipeline}
\end{figure}


\subsection{BridgeDesign から詳細 JSON への変換}


【変換処理】(convert\_simple\_to\_detailed\_json.py)


BridgeDesign から DetailedBridgeJson への変換では、以下の座標計算を行う:


\begin{enumerate}
  \item 主桁配置の X 座標計算
\begin{itemize}
  \item 主桁の総スパン: (num\_girders - 1) × girder\_spacing
  \item X オフセット: (total\_width - 総スパン) / 2
  \item 各主桁の X 座標: x\_offset + i × girder\_spacing
  \par
\end{itemize}
  \item パネル分割の Y 座標計算
\begin{itemize}
  \item Y 座標リスト: [0, panel\_length, 2×panel\_length, ..., bridge\_length]
  \par
\end{itemize}
  \item 横桁の配置計算
\begin{itemize}
  \item 横桁本数: num\_panels - 1(端部を除く)
  \item 縦方向ピッチ: panel\_length
  \item 初期位置: panel\_length
  \par
  \par
\end{itemize}
\end{enumerate}

\begin{figure}[ht]
\centering
\begin{tikzpicture}[
  font=\small,
  arrow/.style={-Latex,thick},
  dim/.style={Latex-Latex,thick},
  girder/.style={draw,thick},
]
  % Deck (plan view)
  \draw[thick] (0,0) rectangle (10,4);
  \node[anchor=north] at (5,0) {X(幅員方向)};
  \node[anchor=west] at (10,2) {Y(橋長方向)};

  % Axes
  \draw[arrow] (0,0) -- (11,0) node[below] {X};
  \draw[arrow] (0,0) -- (0,5) node[left] {Y};
  \draw[arrow] (10.8,4.2) -- (10.8,5.1) node[right] {Z};

  % Origin
  \fill (0,0) circle (1.3pt);
  \node[anchor=north west] at (0,0) {原点(床版左下角)};

  % Girders (4 lines)
  \draw[girder] (2,0) -- (2,4);
  \draw[girder] (4,0) -- (4,4);
  \draw[girder] (6,0) -- (6,4);
  \draw[girder] (8,0) -- (8,4);
  \node[anchor=south] at (5,4) {主桁(例:4本)};

  % Dimensions: x_offset and spacing
  \draw[dim] (0,-0.6) -- node[below]{x\_offset} (2,-0.6);
  \draw[dim] (2,-1.2) -- node[below]{girder\_spacing} (4,-1.2);
  \node[anchor=west] at (0,5.2) {$x = x\_offset + i\,\times\,girder\_spacing$};
\end{tikzpicture}
\caption{座標系(X/Y/Z)と主桁配置の模式図(平面図)}
\label{fig:coord-girders}
\end{figure}


\subsection{詳細 JSON から IFC への変換}


【IFC 生成処理】(convert\_detailed\_json\_to\_ifc.py)


ifcopenshell ライブラリを使用して IFC ファイルを生成する。


主な処理:


\begin{enumerate}
  \item IFC コンテキストのセットアップ
\begin{itemize}
  \item プロジェクト、サイト、建物、スパンの階層構造を作成
  \item 幾何コンテキスト(3D Body)を設定
  \par
\end{itemize}
  \item 床版の生成(Brep)
\begin{itemize}
  \item 2D ポリゴンを押し出して直方体を生成
  \item create\_prism\_from\_2d() で Brep ソリッドを作成
  \par
\end{itemize}
  \item 主桁の生成(SweptSolid)
\begin{itemize}
  \item I 形断面プロファイルを定義(create\_i\_section\_profile())
  \item 断面をパネルごとに押し出し
  \item extrude\_profile\_and\_align() で位置・方向を設定
  \par
\end{itemize}
  \item 横桁の生成(SweptSolid)
\begin{itemize}
  \item I 形断面プロファイルを定義
  \item 各主桁間に配置
  \par
  \par
\end{itemize}
\end{enumerate}


\subsection{生成される IFC 要素}



\subsubsection{IFC エンティティ}


\begin{itemize}
  \item IfcProject: プロジェクト情報
  \item IfcSite: 敷地情報
  \item IfcBuilding: 建物(橋梁)情報
  \item IfcBridgePartTypeEnum.SPAN: 橋梁スパン
  \item IfcBeam: 構造部材(床版、主桁、横桁)
  \par
\end{itemize}


\paragraph{形状表現}


\begin{itemize}
  \item Brep: 床版(境界表現による直方体)
  \item SweptSolid: 主桁・横桁(断面押し出し)
  \par
\end{itemize}


\paragraph{命名規則}


\begin{itemize}
  \item 床版: "Deck"
  \item 主桁: "Girder\_\{桁番号\}\_Seg\_\{セグメント番号\}"
  \item 横桁: "Crossbeam\_Row\{行番号\}\_Gap\{間隔番号\}"
  \par
  \par
\end{itemize}


\clearpage

\section{実行実績と出力例}



\subsection{実行実績の概要}


本システムは既に複数回の設計生成を実行しており、以下の実績がある:


【生成済み設計 JSON】(data/generated\_simple\_bridge\_json/)

\begin{itemize}
  \item 26 件の設計 JSON を生成済み
  \item 対象条件:橋長 30m~50m、幅員 5m~10m
  \par
\end{itemize}

【生成済み IFC ファイル】(data/generated\_ifc/)

\begin{itemize}
  \item 21 件の IFC ファイルを生成済み
  \item BIM/CIM ビューアで 3D モデルとして表示可能
  \par
\end{itemize}


\subsubsection{実行日時の例}

\begin{itemize}
  \item 2025年12月8日 ~ 2026年1月6日にかけて継続的にテスト・改善
  \item 最新実行:2026年1月6日(design\_L50\_B10\_20260106\_163611)
  \par
  \par
\end{itemize}


\subsection{設計出力例(BridgeDesign JSON)}



\subsubsection{実行条件}

\begin{itemize}
  \item 橋長 L = 50 m
  \item 幅員 B = 10 m
  \item モデル: gpt-5-mini
  \item RAG top\_k = 5(各クエリ)
  \par
\end{itemize}

生成された設計値(\texttt{design\_L50\_B10\_20260106\_163611.json}):

\begin{table}[ht]
\centering
\caption{全体寸法(dimensions)}
\begin{tabular}{ll}
\hline
項目 & 値 \\
\hline
橋長 & 50,000 mm (50 m) \\
全幅 & 10,000 mm (10 m) \\
主桁本数 & 4 本 \\
主桁間隔 & 2,667 mm ($\approx$ 2.67 m) \\
パネル長 & 5,000 mm (5 m) \\
パネル数 & 10 \\
\hline
\end{tabular}
\end{table}

\begin{table}[ht]
\centering
\caption{主桁断面(girder\_standard)}
\begin{tabular}{lll}
\hline
項目 & 値 & 備考 \\
\hline
腹板高さ & 2,500 mm & \\
腹板厚 & 16 mm & \\
上フランジ幅 & 600 mm & \\
上フランジ厚 & 40 mm & \\
下フランジ幅 & 800 mm & \\
下フランジ厚 & 50 mm & \\
総桁高 & 2,590 mm & $= 2500 + 40 + 50$ \\
桁高/橋長比 & 1/19.3 & $\approx L/20$ の目安に合致 \\
\hline
\end{tabular}
\end{table}

\begin{table}[ht]
\centering
\caption{横桁断面(crossbeam\_standard)}
\begin{tabular}{lll}
\hline
項目 & 値 & 備考 \\
\hline
桁高 & 2,000 mm & 主桁の約 0.8 倍 \\
腹板厚 & 10 mm & \\
フランジ幅 & 500 mm & 桁高の 1/4 \\
フランジ厚 & 20 mm & \\
\hline
\end{tabular}
\end{table}

\begin{table}[ht]
\centering
\caption{RC床版(deck)}
\begin{tabular}{ll}
\hline
項目 & 値 \\
\hline
床版厚 & 190 mm \\
\multicolumn{2}{l}{参考:連続版式 $d = 30L + 110 = 30 \times 2.667 + 110 \approx 190$ mm} \\
\hline
\end{tabular}
\end{table}


\subsection{RAG 検索結果と設計根拠}



\subsubsection{マルチクエリ RAG 検索の実行結果}


1 回の設計生成で 25 件の参考文献チャンクを取得(代表例):

\begin{table}[ht]
\centering
\caption{RAG 検索結果の例}
\begin{tabular}{clcp{6cm}}
\hline
Rank & ソース & Score & 内容 \\
\hline
1 & 鋼橋設計の基本\_第七章(p.1) & 0.715 & 並列I桁橋の断面例(幅員7.5m、床版21cm、主桁間隔2.3m) \\
2 & 鋼橋設計の基本\_第七章(p.111) & 0.690 & 桁高 $h = L/15$〜$L/20$ の目安、経済的桁高とスパンの関係 \\
17 & 道路橋示方書\_鋼橋・鋼部材編(p.117) & 0.702 & 床版最小厚の算定式(連続版: $d = 30L + 110$) \\
\hline
\end{tabular}
\end{table}


\paragraph{設計根拠(reasoning)の出力例}


"重視した文献箇所:

\begin{itemize}
  \item 主桁桁高の目安(L/20〜L/25)および図7.134のスパン―桁高関係を主に採用し、
  \\ 桁高を決めました(根拠:荷重・たわみ・応力度の総合指標)。
  \item RC床版厚さに関しては道路橋示方書(床版最小厚・連続版の式)を優先して
  \\ 算定しました(根拠:耐久性・最小厚の規定)。
  \item 主桁本数の評価では「主桁間隔の実務例(典型値 ≒3.36m)」と荷重分配・床版
  \\ 剛性の観点、施工上の過大な桁重量回避の観点を秤にかけました。
  \par
\end{itemize}

主桁本数の決定理由:

\begin{itemize}
  \item 候補(3,4,5,6)をoverhang=1.0 mで評価。幅員B=10.0 mから主桁中心間有効幅
  \\ = B − 2·overhang = 8.0 m。
  \item 各候補の主桁間隔:3本→4000 mm、4本→2667 mm、5本→2000 mm、6本→1600 mm
  \item 4本(≈2.667 m)は床版厚を無理なく満足し、かつ主桁間隔が実務例に近く
  \\ 荷重分配・剛性のバランスが良いため採用としました。"
  \par
  \par
\end{itemize}


\subsection{抽出された設計ルール}

Designer は設計値とともに、適用した設計ルールを構造化して出力する。以下は実際に抽出された主要なルール:

\begin{table}[ht]
\centering
\caption{抽出された設計ルール一覧}
\begin{tabular}{clll}
\hline
ID & ルール内容 & 数式・条件 & 根拠 \\
\hline
R1 & 桁高の目安 & $h \approx L/20 \sim L/25$ & 文献 \\
R2 & RC床版最小厚 & $d \geq 30L + 110$ [mm] & 道路橋示方書 \\
R3 & 横桁間隔の制限 & $\leq 20$ m & 文献 \\
R4 & 横桁断面の目安 & 主桁高の 0.8 倍 & 文献 \\
R5 & 主桁間隔の実務目安 & 約 3.36 m & 文献実例 \\
R6 & パネル長の選定 & 4/5/6 m から選定 & 仮定 \\
R7 & 張出し長の目安 & 0.5〜1.5 m & 実務目安 \\
R8 & 横桁の役割 & 所要剛度を満たす程度 & 文献 \\
\hline
\end{tabular}
\end{table}

各ルールには根拠となる RAG ヒットの rank 番号が記録され、設計の透明性を確保している。根拠が文献に見当たらない場合は「仮定」として明記される。


\subsection{生成された IFC ファイル}



生成された IFC ファイルには以下の 3D 要素が含まれる:

\begin{table}[ht]
\centering
\caption{IFC 出力の構成}
\begin{tabular}{llll}
\hline
要素 & 形状表現 & 断面形状 & 要素数 \\
\hline
床版 & Brep & 直方体 & 1 \\
主桁 & SweptSolid & I形 & 40(4桁 $\times$ 10セグメント) \\
横桁 & SweptSolid & I形 & 27(9行 $\times$ 3間隔) \\
\hline
\end{tabular}
\end{table}

\begin{figure}[ht]
\centering
\includegraphics[width=0.92\linewidth]{images/bim_component.png}
\caption{BIM/CIM ビューアでの部材別表示(選択・色分け例)}
\label{fig:bim-component}
\end{figure}

生成された IFC ファイルは BIMvision、Revit、FreeCAD 等の BIM/CIM ソフトウェアで表示可能。


\clearpage

\section{現在の制約と今後の課題}



\subsection{現在の制約事項}



\paragraph{設計面の制約}


\begin{enumerate}
  \item 対象橋種の限定
\begin{itemize}
  \item 現状は鋼プレートガーダー橋(RC床版)のみ
  \item 鋼床版、PC橋、トラス橋などは未対応
  \item これに関しては「将来的には同様に教科書を参照させることで対応しうる」と着地させる
  \par
\end{itemize}
  \item 断面変化への未対応
\begin{itemize}
  \item 全長にわたって同一断面を仮定
  \item 支点部・中央部での断面変化は考慮していない
  \item これはBIMとしてそこまで厳密なものは不要、でにげる
  \par
\end{itemize}
  \item 荷重・応力照査の未実装
\begin{itemize}
  \item 設計値は「目安」レベル
  \item 正式な構造計算による照査は行っていない
  \par
\end{itemize}
  \item 横構・対傾構の未対応
\begin{itemize}
  \item 横桁のみモデル化
  \item 水平ブレーシング等は未対応
  \par
  \par
\end{itemize}
\end{enumerate}


\paragraph{システム面の制約}


\begin{enumerate}
  \item Judge の未実装
\begin{itemize}
  \item 現状はダミー実装(常に OK を返す)
  \item 道路橋示方書に基づく照査機能は未完成
  \par
\end{itemize}
  \item IFC の限定的なサポート
\begin{itemize}
  \item IFC4x3 の限定的なエンティティのみ使用
  \item プロパティセット、材料情報等は未設定
  \par
  \par
\end{itemize}
\end{enumerate}


\subsection{今後の拡張方針}


\paragraph{Judge コンポーネントの完成}


\begin{itemize}
  \item 道路橋示方書に基づく設計照査
  \item 許容応力度法による断面照査
  \item 不適合箇所の指摘と修正提案
  \par
  \par
\end{itemize}


\end{document}
