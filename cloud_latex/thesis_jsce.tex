\documentclass{jsce_template/jjsce}

\usepackage{amsmath}
\usepackage{amsthm}
\usepackage[defaultsups]{newtxtext}
\usepackage[varg]{newtxmath}
\usepackage{bm}
\usepackage[dvipdfmx]{graphicx}
\usepackage[superscript]{cite}
\usepackage{url}
\usepackage{endnotes}
\usepackage[savepos]{zref}
\usepackage[dvipdfmx]{hyperref}
\usepackage{pxjahyper}
\aboveEtitlesep20mm

\usepackage{jsce_template/jjsce-macros}

\begin{document}
\jtitle{LLMを用いた鋼鈑桁橋BIMモデルの自動生成}
\authorlist{%
 \authorentry{阿部 大樹}{Hiroki ABE}{UTokyo}
}
\affiliate[UTokyo]{東京大学工学部社会基盤学科
(\jipcode{113--8656}東京都文京区本郷七丁目3-1)}{hiroki-abe510@g.ecc.u-tokyo.ac.jp}

\begin{abstract}
(後で記載)
\end{abstract}
\begin{Eabstract}
(To be written later)
\end{Eabstract}
\begin{keyword}
BrIM, LLM, RAG, Steel Plate Girder Bridge, IFC, Structural Design Automation
\end{keyword}
\maketitle


\section{はじめに}

(後で記載)


\section{背景および関連研究}

(後で記載)


\section{提案システム}

\subsection{システム概要}

\subsubsection{本システムでできること}

\begin{enumerate}
  \item 最小限の入力から断面設計を自動生成
  \begin{itemize}
    \item 入力:橋長$L$[m]と幅員$B$[m]の2パラメータのみ
    \item 出力:主桁本数,桁高,板厚,床版厚など全断面寸法を自動決定
  \end{itemize}

  \item 設計根拠の明示(RAGによる文献参照)
  \begin{itemize}
    \item 道路橋示方書,鋼橋設計の基本(教科書)から関連条文を自動検索
    \item どの文献のどのページを根拠にしたかをログに記録
    \item 設計者が「なぜその寸法になったか」を確認可能
  \end{itemize}

  \item 設計ルールの構造化抽出
  \begin{itemize}
    \item 「桁高$\approx L/20$」「床版厚$d ≥ 30L + 110$」などの数式を明示
    \item 各ルールに根拠(文献のrank)または「仮定」を付与
  \end{itemize}

  \item 決定論的照査と自動修正ループ(Judge)
  \begin{itemize}
    \item 曲げ・せん断・たわみ・床版厚・腹板幅厚比・横桁配置の6項目を照査
    \item 不合格時はLLMが修正提案(PatchPlan)を生成
    \item 合格するまで自動で設計修正を繰り返す(最大5イテレーション)
  \end{itemize}

  \item IFC形式での3Dモデル出力
  \begin{itemize}
    \item BIM/CIMソフトウェアで表示可能なIFCファイルを生成
    \item 床版,主桁(I形断面),横桁を3Dモデル化
    \item BIMvision,Revit,FreeCAD等で即座に可視化可能
  \end{itemize}
\end{enumerate}


\subsubsection{技術的な特徴}

\begin{itemize}
  \item LLM + RAGによる設計知識の活用
  \begin{itemize}
    \item 道路橋示方書等のPDFから条文を検索し,プロンプトに埋め込み
  \end{itemize}

  \item Structured Outputによる型安全な出力
  \begin{itemize}
    \item Pydanticスキーマで設計値を厳密に定義,バリデーション済み
  \end{itemize}

  \item マルチクエリRAG
  \begin{itemize}
    \item 寸法,主桁配置,主桁断面,床版,横桁の5観点で並行検索
  \end{itemize}

  \item 決定論計算 + LLMハイブリッド照査
  \begin{itemize}
    \item 数式による照査計算は決定論的に実行,修正提案のみLLMが生成
  \end{itemize}

  \item 荷重計算の自動化
  \begin{itemize}
    \item L荷重(p1/p2ルール)に基づく活荷重の内部計算
    \item 桁別の死荷重計算(端桁・中間桁で受け持ち幅が異なる)
  \end{itemize}

  \item 複数候補方式PatchPlan生成
  \begin{itemize}
    \item LLMが3案を生成し,各案を仮適用・評価して最良案を選択
  \end{itemize}

  \item Designer-Judge修正ループ
  \begin{itemize}
    \item 不合格時はPatchPlanを適用し,合格まで自動で繰り返し
  \end{itemize}

  \item 2段階のIFC変換
  \begin{itemize}
    \item 設計JSON → Senkei JSON → IFCの変換パイプライン
  \end{itemize}
\end{itemize}


\subsubsection{対象とする橋梁形式}

\begin{itemize}
  \item 橋梁タイプ:鋼プレートガーダー橋(Steel Plate Girder Bridge)
  \item 床版形式:RC床版合成桁(Reinforced Concrete Deck Composite Girder)
  \item 構造形式:単純桁
  \item 橋長:30m〜70m程度
  \item 幅員:8m〜15m程度
\end{itemize}


\subsubsection{現時点での制約}

\begin{itemize}
  \item 対象は鋼プレートガーダー橋(RC床版)のみ
  \item 断面は全長一定(支点部・中央部の変化なし)
  \item 照査は概略設計レベル(曲げ・せん断・たわみ・床版厚・腹板幅厚比・横桁配置)
  \item 座屈・疲労等の詳細照査は未実装
\end{itemize}


\subsubsection{全体構成}

システムは以下の3層構造で構成される:

\paragraph{ユーザーインターフェース層}
\begin{description}
  \item 統合CLI
  \begin{itemize}
    \item generate:設計JSON生成
    \item run:設計JSON生成 → IFC変換まで一括実行
    \item run\_with\_repair:修正ループ付き実行(各イテレーション保存)
  \end{itemize}
\end{description}

\paragraph{コアロジック層}
\begin{description}
  \item[RAGサブシステム] PDFテキスト抽出,チャンク化,埋め込み生成,ベクトル検索
  \item[Designerサブシステム] プロンプト構築,LLM呼び出し,設計生成
  \item[Judgeサブシステム] 決定論照査計算,PatchPlan生成,修正ループ制御
  \item[IFC変換サブシステム] Simple → Senkei JSON変換,Senkei JSON → IFC変換
\end{description}

\paragraph{外部サービス層}
\begin{description}
  \item[OpenAI API] Responses API(テキスト生成),Embeddings API(ベクトル埋め込み),Structured Output(構造化出力)
\end{description}


\subsection{RAG(検索拡張生成)サブシステム}

\subsubsection{概要}

RAG(Retrieval-Augmented Generation)サブシステムは,道路橋示方書や鋼橋設計の教科書から関連する条文・解説を検索し,DesignerのLLMプロンプトに参考文献として提供する機能を担う.

主な機能:
\begin{itemize}
  \item PDFドキュメントからのテキスト抽出
  \item テキストのチャンク化(分割)
  \item OpenAI Embeddings APIによる埋め込みベクトル生成
  \item コサイン類似度に基づくベクトル検索
\end{itemize}


\subsubsection{対象ドキュメント}

現在RAGで利用しているPDFドキュメント:
\begin{enumerate}
  \item 鋼橋設計の基本\_第一章 概論.pdf:鋼橋の基本概念,設計の考え方
  \item 鋼橋設計の基本\_第四章 鋼橋の設計法.pdf:鋼橋の設計手法,荷重・応力計算
  \item 鋼橋設計の基本\_第六章 床版.pdf:RC床版の設計,厚さ算定式
  \item 鋼橋設計の基本\_第七章 プレートガーダー橋.pdf:プレートガーダー橋の構造・設計
  \item 道路橋示方書\_鋼橋・鋼部材編.pdf:公式の設計基準,条文
\end{enumerate}


\subsubsection{テキスト抽出機能}

PDFからのテキスト抽出には以下の3つの方法を提供している:
\begin{enumerate}
  \item pdfplumber(推奨):テーブル構造の保持に優れる
  \item pypdf:軽量で高速
  \item pymupdf4llm:Markdown形式での出力に対応
\end{enumerate}


\subsubsection{チャンク化とインデックス構築}

チャンク化処理:
\begin{itemize}
  \item 入力:抽出済みテキストファイル(.txt)
  \item 処理:[Page X]マーカーでページごとに分割,各ページテキストを最大800文字ごとに分割,各チャンクにUUID・ソースファイル名・ページ番号を付与
  \item 埋め込みモデル:text-embedding-3-small(OpenAI),1536次元
\end{itemize}


\subsubsection{ベクトル検索機能}

search\_text()関数による検索:
\begin{itemize}
  \item 入力:クエリ文字列,top\_k(返却する上位件数,デフォルト5)
  \item 処理:クエリを埋め込みベクトル化,保存済み埋め込み行列とのコサイン類似度を計算,上位top\_k件を抽出
  \item 出力:SearchResult(チャンクとスコアのペア)のリスト
\end{itemize}


\subsection{Designer(設計生成)サブシステム}

\subsubsection{概要}

Designerサブシステムは,橋長と幅員を入力として受け取り,RAGで取得した参考文献を基にLLM(GPT-5-mini / GPT-5.1)を用いて鋼プレートガーダー橋の断面設計を自動生成する.

主な特徴:
\begin{itemize}
  \item Structured Outputによる型安全な設計データ生成
  \item マルチクエリRAGによる多面的な参考文献検索
  \item 設計ルール(DesignRule)の明示的抽出
  \item 設計根拠(reasoning)の記録
\end{itemize}


\subsubsection{入力仕様}

DesignerInputスキーマ:
\begin{itemize}
  \item bridge\_length\_m: float -- 橋長$L$[m]
  \item total\_width\_m: float -- 幅員$B$[m]
\end{itemize}

入力パラメータは最小限の2つのみとし,その他の設計パラメータ(主桁本数,桁高,板厚など)はLLMがRAGコンテキストを参照して自動決定する.


\subsubsection{出力仕様(BridgeDesignスキーマ)}

BridgeDesign構造:
\begin{itemize}
  \item dimensions:橋長[mm],全幅[mm],主桁本数,主桁間隔[mm],パネル長[mm],パネル数
  \item sections.girder\_standard:腹板高さ・厚さ,上下フランジ幅・厚さ
  \item sections.crossbeam\_standard:桁高,腹板厚,フランジ幅・厚さ
  \item components.deck:床版厚[mm]
\end{itemize}

DesignerOutput構造(LLMからの直接出力):
\begin{itemize}
  \item reasoning: str -- 設計プロセスの思考・判断根拠
  \item rules: list[DesignRule] -- 適用した設計ルール一覧
  \item bridge\_design: BridgeDesign -- 生成された設計
\end{itemize}


\subsubsection{マルチクエリRAG連携}

Designerは設計生成時に,以下の5種類のクエリでRAG検索を実行する:
\begin{enumerate}
  \item 寸法関連(dimensions)
  \item 主桁配置(girder\_layout)
  \item 主桁断面(girder\_section)
  \item RC床版(deck)
  \item 横桁(crossbeam)
\end{enumerate}

各クエリでtop\_k件(デフォルト5件)を取得し,合計最大25チャンクの参考文献をプロンプトに含める.


\subsubsection{設計ルール抽出機能}

Designerは設計値だけでなく,適用した設計ルールを明示的に抽出する.

DesignRuleCategory:
\begin{itemize}
  \item dimensions:橋長・幅員・桁本数・桁間隔・パネル長など全体寸法
  \item girder\_section:主桁断面
  \item deck:RC床版
  \item crossbeam\_section:横桁
  \item other:その他
\end{itemize}


\subsection{Judge(照査・修正提案)サブシステム}

\subsubsection{概要}

Judgeサブシステムは,Designerが生成したBridgeDesignに対して決定論的な照査計算を行い,不合格時にはLLMを用いて修正提案(PatchPlan)を生成する.

主な特徴:
\begin{itemize}
  \item 照査計算は数式ベースの決定論的処理(LLMを使わない)
  \item 修正提案のみLLMが生成するハイブリッドアプローチ
  \item 許可されたアクションの範囲内で修正を提案(安全性確保)
  \item 修正ループにより合格するまで自動で繰り返し
\end{itemize}


\subsubsection{荷重計算}

\paragraph{活荷重計算(L荷重・p1/p2ルール)}

道路橋示方書のB活荷重に基づくL荷重計算を内部で自動実行する.

定数:
\begin{itemize}
  \item $P_{1M} = 10.0$ kN/m$^2$(曲げ照査用p1面圧)
  \item $P_{1V} = 12.0$ kN/m$^2$(せん断照査用p1面圧)
  \item $P_2 = 3.5$ kN/m$^2$(p2面圧,支間80m以下)
  \item 主載荷幅 = 5.5 m,載荷長上限 = 10.0 m
\end{itemize}

計算フロー:
\begin{enumerate}
  \item 載荷長 $D = \min(10.0, L)$
  \item 等価係数 $\gamma = D(2L - D) / L^2$
  \item 等価面圧 $p_{eq} = P_2 + P_1 \times \gamma$
  \item 張り出し幅 = (全幅 - (桁本数-1)$\times$桁間隔) / 2
  \item 受け持ち幅 $b_i$(端桁:張り出し + 桁間隔/2,中間桁:桁間隔)
  \item 実効幅 $b_{eff} = 0.5 \times b_i + 0.5 \times \min(b_i, 5.5)$
  \item 等価線荷重 $w = p_{eq} \times b_{eff}$
  \item 断面力(単純桁)$M = wL^2/8$, $V = wL/2$
\end{enumerate}

\paragraph{死荷重計算(桁別計算)}

死荷重は各主桁の受け持ち幅に応じて個別に計算される.端桁と中間桁で受け持ち幅が異なるため,断面力も異なる.


\subsubsection{照査項目}

Judgeは以下の6項目について照査を実行する:

\begin{enumerate}
  \item \textbf{曲げ応力度照査}
  \begin{itemize}
    \item 計算式:$\sigma = M_{total} \times y / I$
    \item 判定:$\sigma_{actual} / \sigma_{allow} \leq 1.0$(上下フランジ別に評価)
    \item 許容応力度:$\sigma_{allow} = \alpha_{bend} \times f_y$($\alpha_{bend} = 0.6$)
  \end{itemize}

  \item \textbf{せん断応力度照査}
  \begin{itemize}
    \item 計算式:$\tau_{avg} = V_{total} / (t_{web} \times h_{web})$
    \item 判定:$\tau_{avg} / \tau_{allow} \leq 1.0$
    \item 許容応力度:$\tau_{allow} = \alpha_{shear} \times (f_y / \sqrt{3})$($\alpha_{shear} = 0.6$)
  \end{itemize}

  \item \textbf{たわみ照査}
  \begin{itemize}
    \item 計算式:$\delta = 5 \times w_{eq\_live} \times L^4 / (384 \times E \times I)$
    \item 判定:$\delta_{actual} / \delta_{allow} \leq 1.0$
    \item 許容たわみ:道示準拠(支間長により$L/2000$〜$L/500$)
  \end{itemize}

  \item \textbf{床版厚照査}
  \begin{itemize}
    \item 要求床版厚:$d_{required} = 30 \times L_{support} + 110$ [mm]
    \item 判定:deck.thickness $\geq d_{required}$
  \end{itemize}

  \item \textbf{腹板幅厚比照査}
  \begin{itemize}
    \item 要求腹板厚:$t_{web,min} = h_{web} / 130$(SM490の場合)
    \item 判定:$t_{web} \geq t_{web,min}$
  \end{itemize}

  \item \textbf{横桁配置照査}
  \begin{itemize}
    \item 判定:$panel\_length \times num\_panels \approx bridge\_length$(許容誤差0.1\%)
  \end{itemize}
\end{enumerate}


\subsubsection{PatchPlan生成}

不合格時にLLMが生成する修正提案.複数候補方式を採用し,最良案を自動選択.

\paragraph{許可アクション(ALLOWED\_ACTIONS)}
\begin{itemize}
  \item increase\_web\_height:腹板高さを増加($\Delta$: +100, +200, +300, +500 mm)
  \item increase\_web\_thickness:腹板厚を増加($\Delta$: +2, +4, +6 mm)
  \item increase\_top\_flange\_thickness:上フランジ厚を増加
  \item increase\_bottom\_flange\_thickness:下フランジ厚を増加
  \item set\_deck\_thickness\_to\_required:床版厚を要求値に設定
  \item fix\_crossbeam\_layout:num\_panelsまたはpanel\_lengthを修正
\end{itemize}

\paragraph{複数候補方式の処理フロー}
\begin{enumerate}
  \item LLMが3案を生成(異なるアプローチ:桁高重視,フランジ厚重視など)
  \item 各案を仮適用・評価(judge\_v1\_lightweightでmax\_utilをシミュレーション)
  \item improvement(= 現在のmax\_util - シミュレーション後max\_util)が最大の案を選択
\end{enumerate}


\subsubsection{Designer-Judge修正ループ}

run\_with\_repair\_loop()の処理フロー:
\begin{enumerate}
  \item 初期設計生成(Designer)
  \item 照査実行(Judge v1)
  \item pass\_fail == True → 収束・終了
  \item PatchPlanをBridgeDesignに適用(apply\_patch\_plan)
  \item 依存関係ルール適用(apply\_dependency\_rules):例:主桁腹板高さ変更時に横桁高さを連動
  \item 次イテレーションへ(Step 2に戻る)
  \item max\_iterations到達後も不合格 → converged=Falseで終了
\end{enumerate}


\subsection{IFC変換サブシステム}

\subsubsection{概要}

IFC変換サブシステムは,Designer(またはJudge修正後)のBridgeDesign JSONをIFC(Industry Foundation Classes)形式に変換し,BIM/CIMソフトウェアで利用可能な3Dモデルを出力する.

変換は2段階で行われる(推奨パイプライン):
\begin{enumerate}
  \item BridgeDesign JSON → Senkei JSON(中間形式)
  \item Senkei JSON → IFCファイル
\end{enumerate}


\subsubsection{BridgeDesignからSenkei JSONへの変換}

BridgeDesignからSenkei JSONへの変換では,以下の座標計算を行う:
\begin{enumerate}
  \item 主桁配置のX座標計算
  \begin{itemize}
    \item 主桁の総スパン:(num\_girders - 1) $\times$ girder\_spacing
    \item Xオフセット:(total\_width - 総スパン) / 2
    \item 各主桁のX座標:x\_offset + $i \times$ girder\_spacing
  \end{itemize}
  \item パネル分割のY座標計算
  \begin{itemize}
    \item Y座標リスト:[0, panel\_length, 2$\times$panel\_length, ..., bridge\_length]
  \end{itemize}
  \item 横桁の配置計算
  \begin{itemize}
    \item 横桁本数:num\_panels - 1(端部を除く)
    \item 縦方向ピッチ:panel\_length
    \item 初期位置:panel\_length
  \end{itemize}
\end{enumerate}


\subsubsection{Senkei JSONからIFCへの変換}

ifcopenshellライブラリを使用してIFCファイルを生成する.

主な処理:
\begin{enumerate}
  \item IFCコンテキストのセットアップ:プロジェクト,サイト,建物,スパンの階層構造を作成
  \item 床版の生成(Brep):2Dポリゴンを押し出して直方体を生成
  \item 主桁の生成(SweptSolid):I形断面プロファイルを定義,断面をパネルごとに押し出し
  \item 横桁の生成(SweptSolid):I形断面プロファイルを定義,各主桁間に配置
\end{enumerate}


\subsubsection{生成されるIFC要素}

IFCエンティティ:
\begin{itemize}
  \item IfcProject:プロジェクト情報
  \item IfcSite:敷地情報
  \item IfcBuilding:建物(橋梁)情報
  \item IfcBridgePartTypeEnum.SPAN:橋梁スパン
  \item IfcBeam:構造部材(床版,主桁,横桁)
\end{itemize}

形状表現:
\begin{itemize}
  \item Brep:床版(境界表現による直方体)
  \item SweptSolid:主桁・横桁(断面押し出し)
\end{itemize}

命名規則:
\begin{itemize}
  \item 床版:"Deck"
  \item 主桁:"Girder\_\{桁番号\}\_Seg\_\{セグメント番号\}"
  \item 横桁:"Crossbeam\_Row\{行番号\}\_Gap\{間隔番号\}"
\end{itemize}

3Dモデルの座標系:
\begin{itemize}
  \item X軸:橋軸直角方向(幅員方向)
  \item Y軸:橋軸方向(橋長方向)
  \item Z軸:鉛直方向(上向き正)
  \item 原点:床版左下角(X=0, Y=0, Z=床版上面=0)
\end{itemize}


\section{評価実験}

\subsection{実験設定}

(後で記載)

\subsection{評価指標}

(後で記載)

\subsection{結果}

(後で記載)


\section{考察}

(後で記載)


\section{おわりに}

(後で記載)


\appendix
\section*{付録A:主要Pydanticモデル一覧}

\paragraph{Designer関連}
DesignerInput, BridgeDesign, Dimensions, GirderSection, CrossbeamSection, Deck, Sections, Components, DesignerOutput, DesignRule, DependencyRule, DesignerRagLog, RagHit, DesignResult

\paragraph{Judge関連}
JudgeInput, JudgeReport, Utilization, Diagnostics, PatchPlan, PatchAction, RepairIteration, RepairLoopResult, SteelGrade, GoverningCheck

\paragraph{IFC変換関連}
SenkeiJson, Geometry, GirdersGeometry, DeckGeometry, Partition, Crossbeams

\paragraph{RAG関連}
IndexChunk, SearchResult, RagIndex, EmbeddingConfig


\section*{付録B:設計ルールカテゴリ}

DesignRuleCategory(StrEnum):
\begin{itemize}
  \item dimensions:橋長・幅員・桁本数・桁間隔・パネル長など全体寸法
  \item girder\_section:主桁断面(腹板高さ,板厚,フランジ寸法)
  \item deck:RC床版(厚さ)
  \item crossbeam\_section:横桁(桁高,板厚,フランジ寸法)
  \item other:その他
\end{itemize}


\section*{付録C:対応LLMモデル}

LlmModel(StrEnum):
\begin{itemize}
  \item GPT\_5\_MINI = "gpt-5-mini":高速・低コスト,概略設計には十分な精度,デフォルト設定
  \item GPT\_5\_1 = "gpt-5.1":高精度,複雑な設計条件に対応
\end{itemize}


\section*{付録D:参考文献(RAG対象PDF)}

FileNamesUsedForRag:
\begin{enumerate}
  \item 鋼橋設計の基本\_第一章 概論.pdf
  \item 鋼橋設計の基本\_第四章 鋼橋の設計法.pdf
  \item 鋼橋設計の基本\_第六章 床版.pdf
  \item 鋼橋設計の基本\_第七章 プレートガーダー橋.pdf
  \item 道路橋示方書\_鋼橋・鋼部材編.pdf
\end{enumerate}


\begin{thebibliography}{9}
\bibitem{dosisyo}
日本道路協会:道路橋示方書・同解説 I共通編・II鋼橋・鋼部材編,2017. [Japan Road Association: \textit{Specifications for Highway Bridges}, Part I Common, Part II Steel Bridges, 2017.]
\bibitem{textbook}
日本橋梁建設協会:鋼橋設計の基本,2015. [Japan Bridge Association: \textit{Fundamentals of Steel Bridge Design}, 2015.]
\end{thebibliography}

\end{document}
