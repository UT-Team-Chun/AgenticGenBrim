\documentclass{jsce_template/jjsce}

\usepackage{amsmath}
\usepackage{amsthm}
\usepackage[defaultsups]{newtxtext}
\usepackage[varg]{newtxmath}
\usepackage{bm}
\usepackage[dvipdfmx]{graphicx}
\usepackage[superscript]{cite}
\usepackage{url}
\usepackage{tabularx}
\usepackage{endnotes}
\usepackage[savepos]{zref}
\usepackage[dvipdfmx]{hyperref}
\usepackage{pxjahyper}
\aboveEtitlesep20mm

\usepackage{jsce_template/jjsce-macros}

\begin{document}
\jtitle{LLMを用いた鋼プレートガーダー橋BIMモデルの自動生成手法の提案}
\authorlist{%
 \authorentry{阿部 大樹}{Hiroki ABE}{UTokyo}
}
\affiliate[UTokyo]{東京大学工学部社会基盤学科
(\jipcode{113--8656}東京都文京区本郷七丁目3-1)}{hiroki-abe510@g.ecc.u-tokyo.ac.jp}

\begin{abstract}
(後で記載)
\end{abstract}
\begin{Eabstract}
(To be written later)
\end{Eabstract}
\begin{keyword}
BrIM, LLM, RAG, Steel Plate Girder Bridge, IFC, Structural Design Automation
\end{keyword}
\maketitle


\section{はじめに}

我が国では,高度経済成長期に建設された橋梁・トンネル等の社会インフラが老朽化を迎えつつある.建設後50年を超える道路橋の割合は年々増加しており\cite{mlit_infra},インフラの維持管理および更新の重要性がますます高まっている.しかし,土木建設業界は他産業と比較して生産性向上が遅れており,増大する維持管理需要に対して慢性的な人手不足が深刻な課題となっている.こうした状況を受け,国土交通省はi-Construction\cite{iconstruction}をはじめとする施策を通じてICT活用による生産性向上を推進しており,設計・施工・維持管理の各段階においてデジタル技術を活用した業務効率化が強く求められている.

このような生産性向上の取り組みの中核として,BIM/CIM(Building/Construction Information Modeling)の導入が推進されている\cite{bimcim}.BIM/CIMは3次元モデルに属性情報を付与し,設計から維持管理までの一貫したデータ活用を可能とする技術であり,国土交通省は2023年度より小規模工事を除く直轄工事において原則適用を開始した.しかし,3Dモデルの作成には依然として多大な労力と時間を要し,特に橋梁のような複雑な構造物では手作業によるモデリング工数が大きな負担となっている.この工数負担がBIM/CIMの普及を妨げる一因となっており,モデリングの自動化・省力化が喫緊の課題である.

こうした課題に対し,近年急速に発展している大規模言語モデル(LLM)の活用が有望視されている.LLMは自然言語による指示から構造化されたデータを生成する能力を持ち,設計業務の自動化への適用可能性が期待されている.そこで本研究では,LLMを用いて自然言語ベースでBIMモデルを構築できるエージェント型システムを提案する.なお,本研究はLLMによるBIMモデル生成の基礎的な技術開発を範囲とするため、パラメトリックに作成のしやすい鋼プレートガーダー橋(RC床版)に対象を限定する.本システムの特徴は,(1)最小限の入力(橋長・幅員の2パラメータ)から断面設計を自動生成する点,(2)RAG(Retrieval-Augmented Generation: 検索拡張生成)により道路橋示方書等の設計知識を参照し設計根拠を明示する点,(3)決定論的な照査計算とLLMによる自動修正ループにより設計品質を担保する点,(4)IFC形式での出力によりBIM環境との連携を実現する点である.

本論文の構成は以下の通りである.2章では背景および関連研究を述べ,3章で提案システムの詳細を説明する.4章では評価実験の設定と結果を示し,5章で考察を行う.6章で本研究のまとめと今後の課題を述べる.


\section{背景および関連研究}

\subsection{BIM/CIMと橋梁モデリングの現状}

国土交通省は2023年度より小規模工事を除く直轄工事においてBIM/CIM(Building/Construction Information Modeling)の原則適用を開始した\cite{bimcim}.BIM/CIMは設計・施工・維持管理の各段階において3次元モデルと属性情報を一貫して活用する技術であり,建設業の生産性向上に大きく寄与することが期待されている.

BIM/CIMモデルのデータ交換標準としてはIFC(Industry Foundation Classes)が広く採用されている.IFCはbuildingSMART Internationalが策定するオープン標準であり,ベンダー非依存の相互運用性を提供する.特にIFC4x3ではインフラ(橋梁・道路・鉄道等)への対応が強化され,橋梁BIMモデルの標準フォーマットとしての利用が進んでいる\cite{ifc4x3}.

一方で,BIM/CIMの普及には課題も残されている.国土交通省が実施したアンケート調査\cite{mlit_bimcim_questionnaire}によれば,BIM/CIMの導入により「非効率になった」と回答された項目の最多が「モデル作成に手間(時間,費用,人)がかかる」であり,回答者の33\%がこの課題を指摘している.特に橋梁のような複雑な構造物では,3次元モデルの作成に多大な工数を要することがBIM/CIM普及の障壁となっている.本研究はこのモデリング工数の課題に対し,LLMによる自動化アプローチを提案するものである.

\subsection{LLMの建設・土木分野への応用}

近年,大規模言語モデル(LLM)の建設・土木分野への応用に関する研究が活発化している.ここでは,LLMとBIMモデルの連携に関連する先行研究を整理する.
Nithyananthamら\cite{mcp4ifc}は,LLMでIFCデータをMCP(Model Context Protocol)経由で直接操作するフレームワーク「MCP4IFC」を提案した.このフレームワークにより,自然言語による指示からBIMモデルの作成・編集が可能となるが,対象は壁・部屋など建築向けの汎用的な幾何形状操作が中心である.そのため,橋梁設計において不可欠な構造計算や工学的制約に基づいたモデル生成については考慮されていない.
Zhengら\cite{mdpi_bridge_checking}は,LLMを用いて設計仕様書から構造化ルールを抽出し,IFCモデルの適合性をチェックする手法を提案した.仕様書からのルール抽出精度は79.5\%, IFCモデルへの適合性チェック精度は84.4\%を達成している.ただし,この研究は既存モデルに対する仕様書テキストとの形式的な適合性チェックを主眼としており,応力照査などの数値計算を含む設計プロセスの自動化や,モデルそのものをゼロから生成するアプローチとは異なる.
山本ら\cite{yamamoto_genai_ifc}は,GPT-4oを用いて直方体や円柱などの単純形状のIFCファイルを生成する手法を試みている.しかし,IFCファイルは例えばJSON, csvのような世に広く流通している形式ではなく、LLMの学習データに占める割合が小さいことは容易に推察される。実際,この研究ではスキーマ違反などの構文エラーが多発し,単純な形状でさえ生成成功率が30〜60\%程度に留まることが課題として報告されている.これに対し本研究では,LLMの出力を厳密に型定義された構造化データ(JSON/Pydantic)として受け取り,それをpythonプログラムでIFCへ変換する手法をとることで,BIMデータとしての整合性を保証している.

\subsection{RAG(検索拡張生成)による専門知識活用}

箱石ら\cite{hakoishi_bert}は,汎用的な事前学習済みモデルであるBERTが土木分野の専門文書では精度が低下することを示している.GPTやGeminiなどの基盤モデルが大幅に進化した現在でも,引き続きLLMの性質としての専門分野での精度低下には対応が必要であるが,LLMの専門知識対応には大きく分けてfine-tuningとRAGの2つのアプローチが存在する.

fine-tuningは,事前学習済みモデルを特定ドメインのデータで追加学習させ,専門知識を内部パラメータに埋め込む手法である.しかし,基盤モデルの性能が急速に進化していく2026年1月現在において,フラグシップモデルと呼ばれるような最先端のモデルは公開されておらず,fine-tuningできる状態にないことが多い.さらに,fine-tuningには大量の学習データと計算リソースを必要とし,知識の更新時には再学習が必要となることも課題として大きい.

一方,RAG(Retrieval-Augmented Generation: 検索拡張生成)は,外部知識ベースから関連情報を検索し,その情報を基にテキストを生成する手法である\cite{rag_original}.RAGは以下の手順で動作する:(1) 文書を適当な単位(チャンク)に分割し,各チャンクの埋め込みベクトルを事前に計算してインデックスを構築する,(2) 質問が与えられたとき,質問の埋め込みベクトルと類似度の高いチャンクを検索する,(3) 検索されたチャンクをコンテキストとしてLLMに入力し,回答を生成する.あくまで既存モデルに外部のコンテキストを付与して生成させる手法であるため,API経由で最先端のモデルを使用できることに加え,インデックス構築のみで対応可能であり,知識の更新が文書の差し替えで容易に行えることも利点として大きい.

さらに,Soudaniら\cite{soudani_rag_vs_ft}によると,低頻度かつ専門的な知識に対してはRAGがfine-tuningを大幅に上回る性能を示す.道路橋示方書のような専門的な設計規準は学習データ中の出現頻度が低いため,RAGによるアプローチが適していると考えられる.

以上を踏まえ,本研究ではRAGを採用することで,(1) LLMのハルシネーション(事実と異なる出力)を設計規準の参照により抑制する,(2) どの文献のどのページを根拠としたかを記録し設計根拠のトレーサビリティを確保する,という2つの効果を期待している.

\subsection{本研究の位置付け}

\subsubsection{先行研究との差分}

先行研究と本研究の関係を表\ref{tab:related_work_comparison}に整理する.

\begin{table*}[t]
\caption{先行研究と本研究の機能的・技術的比較}
\label{tab:related_work_comparison}
\centering
\small
\begin{tabularx}{\textwidth}{|l|X|X|}
\hline
先行研究 & アプローチと課題 & 本研究の独自性・解決策 \\
\hline
MCP4IFC\cite{mcp4ifc} & 幾何形状の直接操作(建築汎用) & 橋梁に特化し、RAGと推論に基づく\textbf{自律的な設計パラメータ決定} \\
\hline
Zhengら\cite{mdpi_bridge_checking} & 道路橋のBIMモデルに対して、テキストルールとの照合による形式的な適合性チェック & 力学計算による\textbf{工学的照査}に加え、不適合を解消する\textbf{自動修正ループ} \\
\hline
山本ら\cite{yamamoto_genai_ifc} & LLMによるIFCテキスト直接生成(単純形状のみ) & 構文エラーを防ぐため\textbf{JSONでパラメータセットを生成し、pythonでIFCへ変換する}ことで実構造物のBIMを堅牢に生成 \\
\hline
\end{tabularx}
\end{table*}

\subsubsection{本研究の新規性}

本研究の新規性は以下の3点である:

\begin{enumerate}
  \item \textbf{RAGによる設計根拠の明示}:道路橋示方書等から関連条文を検索し,設計値の根拠をトレース可能とする.LLMのハルシネーションを抑制し,設計者による検証を容易にする.

  \item \textbf{決定論照査とLLM修正のハイブリッド}:照査計算(曲げ・せん断・たわみ・床版厚・腹板幅厚比・横桁配置)は数式ベースの決定論的処理で確実に実行し,修正提案のみLLMが生成する.これにより,工学的妥当性を担保しつつLLMの柔軟な修正提案能力を活用する.

  \item \textbf{IFC出力までの一貫パイプライン}:橋長・幅員の2パラメータ入力から,設計生成 → 照査・修正ループ → IFC出力までを自動化する.BIM/CIM環境への直接連携を実現し,モデリング工数を大幅に削減する.
\end{enumerate}


\section{提案システム}

\subsection{システム概要}

本研究で提案するシステムは,橋長$L$[m]と幅員$B$[m]の2パラメータのみを入力として受け取り,鋼プレートガーダー橋(RC床版)の断面設計を自動生成し,IFC形式の3Dモデルとして出力するエージェント型設計支援システムである.本システムの処理は,RAG(検索拡張生成),Designer(設計生成),Judge(照査・修正提案),IFC変換の4つのサブシステムで構成される.

システムの全体的な処理フローを図\ref{fig:system_overview}に示す.まずDesignerがRAGを用いて道路橋示方書等から関連条文を検索し,それらを参照しながら主桁本数,桁高,板厚,床版厚などの断面寸法を決定する.次にJudgeが生成された設計に対して曲げ・せん断・たわみなどの照査計算を実行し,不合格の場合はLLMが修正提案(PatchPlan)を生成する.この照査と修正のループを合格するまで繰り返し,最終的な設計をIFC形式で出力する.

本システムの主な技術的特徴は以下の5点である.第一に,Structured Output機能を活用することで,LLMの出力をPydanticスキーマで厳密に定義された構造化データとして受け取り,型安全性を確保している.第二に,設計要素ごとに異なるクエリでRAG検索を実行するマルチクエリRAGにより,幅広い設計知識を効率的に参照している.第三に,照査計算は数式ベースの決定論的処理で実行し,修正提案のみLLMが生成するハイブリッドアプローチにより,工学的妥当性と柔軟な修正提案能力を両立している.第四に,LLMが複数の修正案を生成し,各案をシミュレーション評価して最良案を選択する複数候補方式により,効率的な設計改善を実現している.第五に,設計JSONから中間形式(Senkei JSON)を経てIFCに変換する2段階パイプラインにより,BIM環境との連携を実現している.

本システムの対象は単純桁構造の鋼プレートガーダー橋(RC床版合成桁)であり,橋長30m〜70m,幅員8m〜15m程度を想定している.なお,現時点では断面は全長一定(支点部・中央部の変化なし)とし,照査は概略設計レベル(曲げ・せん断・たわみ・床版厚・腹板幅厚比・横桁配置)に限定している.座屈・疲労等の詳細照査は今後の課題である.

\begin{figure}[t]
\centering
% TODO: 全体フロー図を挿入
\fbox{\parbox{0.9\linewidth}{\centering 【システム全体フロー図】\\Designer → Judge Loop → IFC変換}}
\caption{提案システムの全体フロー}
\label{fig:system_overview}
\end{figure}


\subsection{RAG(検索拡張生成)サブシステム}

RAGサブシステムは,道路橋示方書や鋼橋設計の教科書から設計に関連する条文・解説を検索し,DesignerのLLMプロンプトに参考文献として提供する機能を担う.本サブシステムは,(1)PDFからのテキスト抽出,(2)テキストのチャンク化,(3)埋め込みベクトル生成,(4)ベクトル検索の4つの処理で構成される.

\subsubsection{対象ドキュメントとテキスト抽出}

本システムでは,鋼プレートガーダー橋の設計に必要な知識を網羅するため,道路橋示方書\cite{dosisyo}および「鋼橋設計の基本」\cite{textbook}から抽出した5種類のPDFドキュメントをRAGの知識ベースとして使用している.具体的には,鋼橋の基本概念と設計の考え方を扱う第一章(概論),荷重・応力計算を扱う第四章(鋼橋の設計法),RC床版の厚さ算定式を扱う第六章(床版),プレートガーダー橋の構造・設計を扱う第七章(プレートガーダー橋),および道路橋示方書の鋼橋・鋼部材編である.

PDFからのテキスト抽出にはpdfplumberライブラリを使用している.pdfplumberは表構造の保持に優れており,示方書のような表形式で設計値が示される文書の処理に適している.抽出処理では,PDFの各ページを順次処理し,ページ終了ごとに「[Page N]」形式のマーカーを挿入することで,後続のチャンク化処理でページ情報を保持できるようにしている.

\subsubsection{チャンク化とインデックス構築}

テキストのチャンク化は2段階で実施される.第一段階では,挿入されたページマーカーに基づいてテキストをページ単位で分割する.第二段階では,各ページのテキストを最大800文字ごとに分割する.この文字数は,LLMのコンテキスト長と検索精度のバランスを考慮して設定した値である.各チャンクには一意なUUID,元PDFのファイル名,ページ番号がメタデータとして付与され,設計根拠のトレーサビリティを確保している.

埋め込みベクトルの生成にはOpenAIのtext-embedding-3-smallモデルを使用している.このモデルは1536次元の埋め込みベクトルを出力し,日本語テキストに対しても高い性能を示す.生成された埋め込みベクトルはL2正規化を施した上で,NumPy配列としてインデックスファイルに保存される.正規化により,ベクトル間の内積がコサイン類似度と等価になり,検索時の計算を効率化できる.数値安定性のため,正規化時には微小値($\varepsilon = 10^{-8}$)を加算している.

\subsubsection{ベクトル検索}

検索時には,クエリ文字列を同じ埋め込みモデルでベクトル化し,保存済みの埋め込み行列との内積(正規化済みのためコサイン類似度と等価)を計算する.上位$k$件(デフォルト$k=5$)の抽出にはNumPyのargpartition関数を使用し,全チャンクのソートを回避することで計算効率を確保している.検索結果は,チャンクのテキスト内容,ソースファイル名,ページ番号,類似度スコアを含むSearchResultオブジェクトのリストとして返される.

インデックスは初回ロード後にメモリ上にキャッシュされ,同一セッション内での複数回の検索において再ロードを回避している.これにより,Designerのマルチクエリ検索(後述)においても高速な応答を実現している.


\subsection{Designer(設計生成)サブシステム}

Designerサブシステムは,橋長$L$と幅員$B$を入力として受け取り,RAGで取得した参考文献を基にLLMを用いて鋼プレートガーダー橋の断面設計を自動生成する.本サブシステムの特徴は,(1)設計要素ごとに異なるクエリでRAG検索を実行するマルチクエリRAG,(2)Pydanticスキーマによる型安全な構造化出力,(3)適用した設計ルールの明示的抽出の3点である.

\subsubsection{入出力仕様}

Designerへの入力は橋長$L$[m]と幅員$B$[m]の2パラメータのみである.この最小限の入力から,LLMがRAGコンテキストを参照しながら,主桁本数,桁間隔,桁高,板厚,床版厚などの全断面寸法を自動決定する.

出力はBridgeDesignと呼ぶPydanticモデルで定義される構造化データである.BridgeDesignは,全体寸法を格納するDimensions(橋長,全幅,主桁本数,主桁間隔,パネル長,パネル数),断面諸元を格納するSections(主桁のI形断面としての腹板高さ・厚さ,上下フランジ幅・厚さ,横桁の断面諸元),および構成要素を格納するComponents(床版厚)の3つのサブモデルで構成される.すべての寸法はmm単位で記録され,Pydanticによるバリデーションにより数値範囲や型の整合性が保証される.

LLMからの直接出力はDesignerOutputモデルで受け取る.このモデルには,BridgeDesignに加えて,設計プロセスの思考・判断根拠を記述したreasoning(文字列),および適用した設計ルールの一覧であるrules(DesignRuleのリスト)が含まれる.これにより,生成された設計値だけでなく,その根拠も記録される.

\subsubsection{マルチクエリRAG}

Designerは設計生成時に,設計要素ごとに異なる5種類のクエリでRAG検索を実行する.第一のクエリは寸法関連(dimensions)であり,「鋼プレートガーダー橋 橋長$L$m 幅員$B$m 桁配置 主桁本数 桁間隔 パネル長」といった内容で,主桁本数や桁間隔の決定に必要な知識を検索する.第二のクエリは主桁配置(girder\_layout)であり,主桁間隔と幅員の関係,幅員と主桁本数の実例を検索する.第三のクエリは主桁断面(girder\_section)であり,経済的桁高の目安($h/L$比),腹板厚さ,フランジ寸法を検索する.第四のクエリはRC床版(deck)であり,床版厚さの算定式や最小床版厚を検索する.第五のクエリは横桁(crossbeam)であり,横桁の設計方法を検索する.

各クエリで上位5件のチャンクを取得するため,合計最大25チャンクの参考文献がプロンプトに含まれる.プロンプトでは,これらのチャンクにrank番号(1〜25)を付与し,LLMが設計値を決定する際にどのチャンクを根拠としたかをsource\_hit\_ranksとして記録させる.根拠となる文献が見つからない場合は,source\_hit\_ranksを空リストとし,notesに「仮定」「実務目安」と記述させることで,根拠の有無を明確に区別している.

\subsubsection{設計ルール抽出}

Designerは設計値だけでなく,適用した設計ルールを明示的に抽出する.各DesignRuleには,ルールID(R1, R2, ...),カテゴリ(dimensions, girder\_section, deck, crossbeam\_section, other),日本語要約(1〜3文),条件式(例:$h_{\mathrm{web}} \approx L/20 \sim L/25$),LaTeX表現,影響するフィールドパス,根拠となるRAGヒットのrank番号,補足が含まれる.

さらに,部材間の依存関係を表すDependencyRuleも抽出する.例えば,横桁高さが主桁高さの80\%程度とする場合,対象フィールド(sections.crossbeam\_standard.total\_height),参照フィールド(sections.girder\_standard.web\_height),係数(0.8)を記録する.この依存関係ルールは,Judge修正ループ(後述)において主桁高さが変更された際に横桁高さを自動更新するために使用される.

\subsubsection{Structured OutputによるLLM呼び出し}

LLMの呼び出しにはOpenAIのResponses APIとStructured Output機能を使用している.Structured Outputは,Pydanticスキーマを指定することでLLMの出力を型安全な構造化データとして受け取る機能である.具体的には,DesignerOutputモデルをスキーマとして指定し,LLMがこのスキーマに準拠したJSONを生成するよう制約する.OpenAI APIが自動でバリデーションとPydanticインスタンス化を行うため,出力の型安全性が保証される.この方式により,山本ら\cite{yamamoto_genai_ifc}が報告したIFCテキスト直接生成時のスキーマ違反や構文エラーの問題を回避している.


\subsection{Judge(照査・修正提案)サブシステム}

Judgeサブシステムは,Designerが生成したBridgeDesignに対して決定論的な照査計算を行い,不合格時にはLLMを用いて修正提案(PatchPlan)を生成する.本サブシステムの特徴は,(1)照査計算を数式ベースの決定論的処理で実行しLLMを使用しない点,(2)修正提案のみLLMが生成するハイブリッドアプローチ,(3)許可されたアクションの範囲内で修正を提案することで安全性を確保している点,(4)照査と修正のループにより合格するまで自動で繰り返す点である.

\subsubsection{荷重計算}

照査計算に先立ち,死荷重と活荷重の断面力を計算する.活荷重には道路橋示方書\cite{dosisyo}のB活荷重に基づくL荷重を適用する.L荷重は,部分載荷を全スパン等分布荷重に換算する等価係数$\gamma$を用いて計算される.

まず載荷長$D$を$D = \min(10.0, L)$で定め,等価係数を式(\ref{eq:gamma})で算出する.
\begin{equation}
\gamma = \frac{D(2L - D)}{L^2}
\label{eq:gamma}
\end{equation}
次に,等価面圧$p_{\mathrm{eq}}$を式(\ref{eq:peq})で計算する.
\begin{equation}
p_{\mathrm{eq}} = p_2 + p_1 \times \gamma
\label{eq:peq}
\end{equation}
ここで,$p_2 = 3.5$ kN/m$^2$(支間80m以下の場合),$p_1 = 10.0$ kN/m$^2$(曲げ照査用)または$p_1 = 12.0$ kN/m$^2$(せん断照査用)である.

各主桁の受け持ち幅$b_i$は,端桁と中間桁で異なる.張り出し幅$c$を$c = (B - (n_g - 1) \times s_g) / 2$($B$: 全幅,$n_g$: 主桁本数,$s_g$: 主桁間隔)として,端桁は$b_i = c + s_g / 2$,中間桁は$b_i = s_g$となる.さらに,主載荷幅5.5mを考慮した実効幅$b_{\mathrm{eff}}$を式(\ref{eq:beff})で計算する.
\begin{equation}
b_{\mathrm{eff}} = 0.5 \times b_i + 0.5 \times \min(b_i, 5.5)
\label{eq:beff}
\end{equation}
活荷重による等価線荷重$w_{\mathrm{live}}$は$w_{\mathrm{live}} = p_{\mathrm{eq}} \times b_{\mathrm{eff}}$となり,単純桁として曲げモーメント$M_{\mathrm{live}} = w_{\mathrm{live}} L^2 / 8$,せん断力$V_{\mathrm{live}} = w_{\mathrm{live}} L / 2$を算出する.

死荷重については,床版のRC重量と主桁の鋼重量を各主桁の受け持ち幅に応じて個別に計算する.床版による線荷重$w_{\mathrm{deck}}$は$w_{\mathrm{deck}} = \gamma_c \times t_d \times b_i$($\gamma_c$: コンクリート単位体積重量,$t_d$: 床版厚)であり,主桁の鋼重量$w_{\mathrm{steel}}$は断面積と鋼の単位体積重量から算出する.

\subsubsection{照査項目}

Judgeは6項目の照査を実行し,各項目について需要と許容値の比(utilization ratio)を算出する.すべての項目でutilization ratioが1.0以下であれば合格とする.

第一に,曲げ応力度照査を行う.死荷重と活荷重による全曲げモーメント$M_{\mathrm{total}}$に対し,上縁および下縁の曲げ応力度$\sigma$を式(\ref{eq:sigma})で計算する.
\begin{equation}
\sigma = \frac{M_{\mathrm{total}} \times y}{I}
\label{eq:sigma}
\end{equation}
ここで,$y$は中立軸から上縁または下縁までの距離,$I$は断面二次モーメントである.許容曲げ応力度$\sigma_{\mathrm{allow}}$は$\sigma_{\mathrm{allow}} = \alpha_{\mathrm{bend}} \times f_y$($\alpha_{\mathrm{bend}} = 0.6$,$f_y$: 降伏点)とし,上縁と下縁それぞれについてutilization ratioを評価する.

第二に,せん断応力度照査を行う.平均せん断応力度$\tau_{\mathrm{avg}}$を式(\ref{eq:tau})で計算する.
\begin{equation}
\tau_{\mathrm{avg}} = \frac{V_{\mathrm{total}}}{t_{\mathrm{web}} \times h_{\mathrm{web}}}
\label{eq:tau}
\end{equation}
許容せん断応力度$\tau_{\mathrm{allow}}$は$\tau_{\mathrm{allow}} = \alpha_{\mathrm{shear}} \times (f_y / \sqrt{3})$($\alpha_{\mathrm{shear}} = 0.6$)である.

第三に,たわみ照査を行う.活荷重によるたわみ$\delta$を式(\ref{eq:delta})で計算する.
\begin{equation}
\delta = \frac{5 w_{\mathrm{live}} L^4}{384 E I}
\label{eq:delta}
\end{equation}
許容たわみ$\delta_{\mathrm{allow}}$は道路橋示方書に基づき,支間長$L \leq 10$mで$L/2000$,$10$m$< L \leq 40$mで$L^2/20000$,$L > 40$mで$L/500$とする.

第四に,床版厚照査を行う.要求床版厚$t_{\mathrm{req}}$を$t_{\mathrm{req}} = \max(30 L_{\mathrm{support}} + 110, 160)$[mm]($L_{\mathrm{support}}$: 床版支間[m])で算出し,設計床版厚がこれ以上であることを確認する.

第五に,腹板幅厚比照査を行う.SM490鋼の場合,要求腹板厚$t_{\mathrm{web,min}}$を$t_{\mathrm{web,min}} = h_{\mathrm{web}} / 130$で算出し,設計腹板厚がこれ以上であることを確認する.

第六に,横桁配置照査を行う.パネル長とパネル数の積が橋長とほぼ一致すること(許容誤差1.0mm以内),およびパネル長が20m以下であることを確認する.

\subsubsection{PatchPlan生成}

照査で不合格となった場合,LLMが修正提案(PatchPlan)を生成する.PatchPlanは,許可されたアクションの組み合わせで構成される.許可されるアクションは,腹板高さの増加(+100, +200, +300, +500mm),腹板厚の増加(+2, +4, +6mm),上下フランジ厚の増加(+2, +4, +6mm),上下フランジ幅の増加(+50, +100mm),床版厚を要求値に設定,横桁配置の修正の8種類である.各PatchPlanには最大3件のアクションを含めることができる.

LLMへのプロンプトでは,照査結果(各項目のutilization ratio,支配的な照査項目)と現在の設計値を提示し,max\_util(全項目のutilization ratioの最大値)を1.0以下(できれば0.98以下)にするための修正案を要求する.プロンプトには判断の方針も示しており,例えば曲げが支配的な場合はフランジ厚または腹板高さの増加を,たわみが支配的な場合は断面二次モーメントを増やす方向(腹板高さ増加)を,せん断が支配的な場合は腹板厚の増加を優先するよう指示している.

本システムでは複数候補方式を採用している.LLMが3つの異なるアプローチ(例:腹板高さ重視,フランジ厚重視,バランス型)による修正案を同時に生成し,各案を仮適用して照査計算をシミュレーションする.そして,改善度(improvement = 現在のmax\_util - シミュレーション後のmax\_util)が最大となる案を選択する.この方式により,LLMの提案の不確実性を軽減し,効率的な設計改善を実現している.

\subsubsection{修正ループ}

Designer-Judge修正ループの処理フローを以下に示す.まずDesignerが初期設計を生成し,Judgeが照査を実行する.合格(max\_util $\leq$ 1.0かつ横桁配置OK)であれば終了する.不合格の場合,LLMがPatchPlanを生成し,設計に適用する.さらに,Designerが抽出した依存関係ルール(例:横桁高さが主桁高さの80\%)を適用し,部材間の整合性を維持する.その後,再度Judgeが照査を実行する.このループを合格するまで,または最大イテレーション数(デフォルト5回)に達するまで繰り返す.

修正ループの結果はRepairLoopResultとして記録される.これには,収束フラグ(converged),全イテレーションの履歴(各イテレーションの設計値,照査結果,適用したPatchPlan),最終設計(final\_design),最終照査結果(final\_report),およびRAG検索ログ(rag\_log)が含まれる.これにより,設計改善のプロセス全体を追跡可能としている.


\subsection{IFC変換サブシステム}

IFC変換サブシステムは,Designerが生成した(またはJudge修正後の)BridgeDesign JSONをIFC(Industry Foundation Classes)形式に変換し,BIM/CIMソフトウェアで利用可能な3Dモデルを出力する.変換は2段階のパイプラインで行われる.第一段階でBridgeDesign JSONを中間形式であるSenkei JSONに変換し,第二段階でSenkei JSONをIFCファイルに変換する.この2段階方式により,橋梁BIMの既存エコシステムとの接続性を確保しつつ,将来的な拡張にも対応可能な設計としている.

\subsubsection{BridgeDesignからSenkei JSONへの変換}

Senkei JSONは,橋梁の3次元形状を線形(Senkei)ベースで定義する中間形式である.BridgeDesignの断面諸元から,主桁・横桁・床版の3次元位置を具体的な座標値として算出する.

主桁の配置では,まず全主桁の総スパン$(n_g - 1) \times s_g$を計算し,全幅との差から張り出し幅(Xオフセット)を求める.各主桁のX座標(幅員方向位置)はXオフセット$+ i \times s_g$($i = 0, 1, ..., n_g - 1$)となる.さらに,各主桁について6本の線形を定義する.これらは上フランジの左端・中央・右端,下フランジの左端・中央・右端に対応し,フランジ幅と腹板高さに基づいてY座標(横方向)とZ座標(鉛直方向)を決定する.

橋軸方向(X座標)の分割は,パネル長に基づいて行う.始端(S1, $x = 0$)から終端(E1, $x = L$)まで,パネル長ごとに断面位置(C1, C2, ...)を設定する.これにより,主桁は複数のブロック(パネル)に分割され,各ブロックが独立したIFC要素として生成される.

横桁は,隣接する主桁間を接続する形で配置される.横桁の縦方向位置はパネル位置(C1, C2, ...)に一致し,両端部を除く$n_{\mathrm{panels}} - 1$箇所に配置される.各横桁の断面諸元(高さ,腹板厚,フランジ寸法)はBridgeDesignのcrossbeam\_standardから取得する.

床版は,橋梁の四隅を頂点とする矩形として定義される.厚さはBridgeDesignのdeck.thicknessから取得し,床版上面をZ=0とする座標系を採用している.

\subsubsection{Senkei JSONからIFCへの変換}

第二段階では,ifcopenshellライブラリを使用してSenkei JSONをIFCファイルに変換する.ifcopenshellはPythonバインディングを提供するIFC操作ライブラリであり,IFC2x3およびIFC4x3の両方に対応している.

変換処理では,まずIFCの階層構造(IfcProject → IfcSite → IfcBuilding → IfcBuildingStorey)をセットアップする.次に,Senkei JSONの各要素を対応するIFCエンティティに変換する.

床版はIfcPlateエンティティとして生成し,形状表現にはBrep(境界表現)を使用する.2Dポリゴン(四隅の座標)を床版厚さ分だけ押し出すことで直方体を生成する.

主桁はIfcBeamエンティティとして生成する.形状表現にはSweptSolid(断面押し出し)を使用し,I形断面プロファイル(IfcIShapeProfileDef)を定義した上で,各パネル長さ分だけ押し出す.各主桁パネルには「G\{桁番号\}B\{ブロック番号\}W」(ウェブ),「G\{桁番号\}B\{ブロック番号\}UF」(上フランジ),「G\{桁番号\}B\{ブロック番号\}LF」(下フランジ)といった命名規則で一意な名前を付与する.

横桁も同様にIfcBeamエンティティとして生成し,I形断面の押し出しで形状を表現する.横桁の押し出し方向は主桁と直交する方向(幅員方向)となる.命名規則は「CB\_G\{桁i\}\_G\{桁i+1\}\_C\{断面番号\}」である.

材料属性(SM490A等)はIfcMaterialエンティティとして定義し,各構造要素に関連付ける.これにより,BIMビューワーでの属性表示や数量計算が可能となる.

\subsubsection{座標系と出力形式}

3Dモデルの座標系は,X軸を橋軸方向(橋長方向),Y軸を橋軸直角方向(幅員方向),Z軸を鉛直方向(上向き正)とする.原点は床版上面の角(X=0, Y=0, Z=0)に設定している.この座標系は道路橋の一般的な表現に準拠している.

生成されるIFCファイルはIFC2x3形式またはIFC4x3形式で出力可能である.IFC4x3ではインフラ向けの拡張(IfcBridgePartTypeEnum等)が利用可能であり,橋梁BIMとしてより適切な表現が可能となる.出力されたIFCファイルは,BIMvision,Revit,FreeCADなどの汎用BIMビューワーで即座に可視化・検証できる.


\section{評価実験}

\subsection{実験設定}

(後で記載)

\subsection{評価指標}

(後で記載)

\subsection{結果}

(後で記載)


\section{考察}

(後で記載)


\section{おわりに}

(後で記載)


\appendix
\section*{付録A:主要Pydanticモデル一覧}

\paragraph{Designer関連}
DesignerInput, BridgeDesign, Dimensions, GirderSection, CrossbeamSection, Deck, Sections, Components, DesignerOutput, DesignRule, DependencyRule, DesignerRagLog, RagHit, DesignResult

\paragraph{Judge関連}
JudgeInput, JudgeReport, Utilization, Diagnostics, PatchPlan, PatchAction, RepairIteration, RepairLoopResult, SteelGrade, GoverningCheck

\paragraph{IFC変換関連}
SenkeiJson, Geometry, GirdersGeometry, DeckGeometry, Partition, Crossbeams

\paragraph{RAG関連}
IndexChunk, SearchResult, RagIndex, EmbeddingConfig


\section*{付録B:設計ルールカテゴリ}

DesignRuleCategory(StrEnum):
\begin{itemize}
  \item dimensions:橋長・幅員・桁本数・桁間隔・パネル長など全体寸法
  \item girder\_section:主桁断面(腹板高さ,板厚,フランジ寸法)
  \item deck:RC床版(厚さ)
  \item crossbeam\_section:横桁(桁高,板厚,フランジ寸法)
  \item other:その他
\end{itemize}


\section*{付録C:対応LLMモデル}

LlmModel(StrEnum):
\begin{itemize}
  \item GPT\_5\_MINI = "gpt-5-mini":高速・低コスト,概略設計には十分な精度,デフォルト設定
  \item GPT\_5\_1 = "gpt-5.1":高精度,複雑な設計条件に対応
\end{itemize}


\section*{付録D:参考文献(RAG対象PDF)}

FileNamesUsedForRag:
\begin{enumerate}
  \item 鋼橋設計の基本\_第一章 概論.pdf
  \item 鋼橋設計の基本\_第四章 鋼橋の設計法.pdf
  \item 鋼橋設計の基本\_第六章 床版.pdf
  \item 鋼橋設計の基本\_第七章 プレートガーダー橋.pdf
  \item 道路橋示方書\_鋼橋・鋼部材編.pdf
\end{enumerate}


\begin{thebibliography}{99}
\bibitem{dosisyo}
日本道路協会:道路橋示方書・同解説 I共通編・II鋼橋・鋼部材編,2017.
\bibitem{textbook}
日本橋梁建設協会:鋼橋設計の基本,2015.
\bibitem{mlit_infra}
国土交通省:社会資本の老朽化の現状と将来,\url{https://www.mlit.go.jp/sogoseisaku/maintenance/02research/02_01.html}(参照 2025-01-26).
\bibitem{iconstruction}
国土交通省:i-Construction,\url{https://www.mlit.go.jp/tec/tec_tk_000028.html}(参照 2025-01-26).
\bibitem{bimcim}
国土交通省:BIM/CIM,\url{https://www.mlit.go.jp/tec/tec_tk_000037.html}(参照 2025-01-26).
\bibitem{ifc4x3}
buildingSMART International: IFC 4.3 (ISO 16739-1:2024) Infrastructure Extensions, \url{https://www.buildingsmart.org/standards/bsi-standards/industry-foundation-classes/} (accessed 2025-01-26).
\bibitem{mlit_bimcim_questionnaire}
国土交通省:BIM/CIM原則適用に関するアンケート調査結果,\url{https://www.mlit.go.jp/tec/content/001757200.pdf}(参照 2025-01-26).
\bibitem{mcp4ifc}
Nithyanantham, S., et al.: MCP4IFC: A Framework for Automated IFC Data Manipulation Using Large Language Models and Model Context Protocol, arXiv:2511.05533, 2025.
\bibitem{mdpi_bridge_checking}
Zheng, Z., et al.: Automated Checking of Highway Bridge BIM Models Using Large Language Models, Buildings, Vol.15, No.19, 3465, 2025.
\bibitem{yamamoto_genai_ifc}
山本 敦大,緒方 陸,藤井 純一郎,山本 一浩:生成 AI による単純形状 3 次元データモデルの対話的な生成手法の一考察,AI・データサイエンス論文集,6巻1号,pp.96--106,2025.
\bibitem{hakoishi_bert}
箱石 健太, 一言 正之, 菅田 大輔:土木分野における 事前学習モデル BERT による精度検証,土木学会論文集,Vol.79,No.22,22-22042,2023.
\bibitem{rag_original}
Lewis, P., et al.: Retrieval-Augmented Generation for Knowledge-Intensive NLP Tasks, Advances in Neural Information Processing Systems, Vol.33, pp.9459--9474, 2020.
\bibitem{soudani_rag_vs_ft}
Soudani, H., et al.: Fine-Tuning vs. Retrieval-Augmented Generation for Less Popular Knowledge, arXiv:2403.01432, 2024.
\end{thebibliography}

\end{document}
